% ~ ........................................
% ~ Section A
% ~ ======================================== 
\begin{center}
   \large
   \textbf{SECTION A (20 marks)}\\
   \vspace{5mm}
   \normalsize
   \textit{(Attempt \textbf{all} questions from this \textbf{Section})}
\end{center}
\par
\noindent
% ~  Question 1
% ~  ----------
\textbf{Question 1}\\
Choose the correct answers to the questions from the given options. \hfill [5] \\
(Do not copy the questions, write the correct answers only.)
\par
\vspace{2mm}
\begin{enumerate}[label=(\roman*)]

    % ------- Question 1.i ----------------------------------
    % ~   (i) - a
    % ~       MLQB - 33/18
    \item Nucleus is located at:

        \begin{enumerate}[label=(\alph*)]
            \setlength\itemsep{0em}
            \item Center of cytoplasm
            \item Equator of the cell
            \item Between the cell membrane and cytoplasm
            \item Between nucleolus and cell membrane
        \end{enumerate}

    % ------- Question 1.ii ----------------------------------
    % ~   (ii) - a
    % ~       MLQB - 65/2
    \item Drooping of leaves in \textit{Mimosa pudica} plant is due to:

        \begin{enumerate}[label=(\alph*)]
            \setlength\itemsep{0em}
            \item Change in turgor pressure
            \item Imbibition
            \item Plasmolysis
            \item Diffusion
        \end{enumerate}

    % ------- Question 1.iii ----------------------------------
    % ~   (iii) - b
    % ~       MLQB - 89/3
    \newpage
    \item The rate of transpiration increases with:

        \begin{enumerate}[label=(\alph*)]
            \setlength\itemsep{0em}
            \item Increase in humidity
            \item Increase in wind velocity
            \item Reduced light intensity
            \item Increase in the CO\textsubscript{2} level
        \end{enumerate}

    % ------- Question 1.iv ----------------------------------
    % ~   (iv) - b
    % ~       MLQB - 1/1
    \item Which statement is true for both chromosomes and genes?

        \begin{enumerate}[label=(\alph*)]
            \setlength\itemsep{0em}
            \item Each codes for a specific protein
            \item Each may be copied and passed on in mitosis
            \item Each may be either dominant or recessive
            \item Each may exist as two or more alleles
        \end{enumerate}

    % ------- Question 1.v ----------------------------------
    % ~   (v) - d
    % ~       MLQB - 110/10
    \item In the guard cells, if the water content fall short, \rule{2cm}{0.15mm} occurs 
        which makes the cells \rule{2cm}{0.15mm}.

        \begin{enumerate}[label=(\alph*)]
            \setlength\itemsep{0em}
            \item Endosmosis, turgid 
            \item Exomosis, turgid 
            \item Endosmosis, flaccid 
            \item Exosmosis, flaccid 
        \end{enumerate}

\end{enumerate}

% ======================================================================
% ~  Question 2
% ~  ----------
\par
\noindent
\textbf{Question 2}\\
\begin{enumerate}[label=(\roman*)]

    % ~   (2.i) 
    % ~     a) Chloroplast
    % ~     b) Carbon-dioxide
    % ~     c) Cytokinins
    % ~     d) Xerophyte
    % ~     e) Root hair
    % ~        
    \item Name the following: \hfill [5]
        \begin{enumerate}[label=(\alph*)]
            \setlength\itemsep{0em}
            \item The principal site in a green leaves for photosynthesis.
            \item The gas which helps in building up of carbohydrates in plants.
            \item Basic hormones that stimulate cell division.
            \item Type of plants having sunken stomata.
            \item Part of plant which absorbs water from the soil.
        \end{enumerate}

\end{enumerate}

