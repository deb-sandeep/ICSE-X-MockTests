% ~ ........................................
% ~ Section B
% ~ ======================================== 
\newpage
\begin{center}
   \large
   \textbf{SECTION B (10 marks)}\\
   \vspace{5mm}
   \normalsize
   \textit{(Attempt \textbf{any one} questions from this \textbf{Section})}
\end{center}
\par

% ======================================================================
% ~  Question 3
% ~  ----------
\noindent
\textbf{Question 3}
\begin{enumerate}[label=(\roman*)]
 
    % ~   (3.i) K+ is larger.
    % ~     MLQB - 38/104
    % ~
    \item Which is larger \ce{Na+} or \ce{K+}? Why? \hfill [2]

    % ~   (3.ii) H3O+ - Coordinate bond between H+ and H2O
    % ~
    \item Explain how a hydronium ion is formed. \hfill [2]

    % ~   (3.iii) 
    % ~     (a) Electrovalent bond
    % ~     (b) Sulphuric or Sulphurous acid
    % ~     (c) Sulphudic acid
    % ~
    \item One word or technical terms: \hfill [3]
        \begin{enumerate}[label=(\alph*)]
            \item A bond formed by the transfer of electron.
            \item Two dibasic acids containing sulphur.
            \item Acid used in electrolysis of water.
        \end{enumerate}

    % ~   (3.iv) 
    % ~     (a) MLQB - 79/63.1
    % ~        Fe + H2SO4 (dil) -> FeSO4 + H2 ^ 
    % ~     (b) MLQB - 78/11
    % ~        Pb(NO3)w + H2SO4 (dil) -> PbSO4 v + 2HNO3 
    % ~     (c) MLQB - 158/69.1
    % ~        Cathode : Ni2+ + 2e- -> Ni
    % ~        Anode : Ni - 2e- -> Ni2+
    % ~
    \item  \hfill [3]
        \begin{enumerate}[label=(\alph*)]
            \item Write the equation for laboratory preparation of Iron (II) sulphate
                from Iron.
            \item Write balanced chemical equation for the reaction of Lead Nitrate 
                and dilute Sulphuric acid.
            \item Write equations for reactions taking place at anode and cathode during
                the electrolysis of acidified nickel sulphate solution with nickel
                electrode.
        \end{enumerate}

\end{enumerate}

% ======================================================================
% ~  Question 4
% ~  ----------
\noindent
\textbf{Question 4}
\begin{enumerate}[label=(\roman*)]

    % ~   (4.i) Tendency to loose electron decreases from left to right.
    % ~      MLQB - 38/101
    \item The reducing power of elements decreases from left to right in 
        a period. Why? \hfill [2]

    % ~   (4.ii) Increase of effective nuclear charge.
    % ~
    \item A cation is smaller than the atom from which it is formed. Why? \hfill [2]

    % ~   (4.iii) MLQB - 161/87
    % ~
    \item Give three points of difference between an electrolytic cell and 
        an electrochemical cell. \hfill [3]

    % ~   (4.iv) MLQB - 164/105
    % ~     (a) Aesthetics, Corrosion resistance
    % ~     (b) Metal is discharged at cathode 
    % ~     (c) Facilitiating migration of metal ions and deposition at cathode
    % ~
    \item With reference to electroplating answer the following questions: \hfill [3]
        \begin{enumerate}[label=(\alph*)]
            \item Why are articles electroplated?
            \item Why the article to be electrolated is made a cathode?
            \item Why is a direct current used?
        \end{enumerate}

\end{enumerate}

