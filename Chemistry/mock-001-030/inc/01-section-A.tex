% ~ ........................................
% ~ Section A
% ~ ======================================== 
\begin{center}
   \large
   \textbf{SECTION A (20 marks)}\\
   \vspace{5mm}
   \normalsize
   \textit{(Attempt \textbf{all} questions from this \textbf{Section})}
\end{center}
\par
\noindent
% ~  Question 1
% ~  ----------
\textbf{Question 1}\\
Choose the correct answers to the questions from the given options. \hfill [5]\\
(Do not copy the questions, write the correct answers only.)
\par
\vspace{2mm}
\begin{enumerate}[label=(\roman*)]

    % ------- Question 1.i ----------------------------------
    % ~   (i) - a. Ionization energy increases across period and decreases 
    % ~         the group.
    % ~       MLQB - 30/8
    \item Arrange \ce{K, Cl, Na, S, Si} in increasing order of ionization energy.

        \begin{multicols}{2}
        \begin{enumerate}[label=(\alph*)]
            \setlength\itemsep{0em}
            \item K < Na < Si < S < Cl
            \item K < Na < Si < Cl < S
            \item Na < S < Cl < K < Si
            \item Si < S < Cl < Na
        \end{enumerate}
        \end{multicols}

    % ------- Question 1.ii ----------------------------------
    % ~   (ii) - b
    % ~       MLQB - 50/3
    \item Covalent bond is rigid and directional. It is responsible for \rule{1.5cm}{0.15mm}. 

        \begin{enumerate}[label=(\alph*)]
            \setlength\itemsep{0em}
            \item rigidity of the molecule
            \item definite shape of the molecule
            \item fluidity of the molecule
            \item None of these
        \end{enumerate}

    % ------- Question 1.iii ----------------------------------
    % ~   (iii) - a
    % ~       MLQB - 70/6
    \newpage
    \item  The salt prepared by the method of direct combination is \rule{1.5cm}{0.15mm}.

        \begin{multicols}{2}
        \begin{enumerate}[label=(\alph*)]
            \setlength\itemsep{0em}
            \item Iron (III) Chloride
            \item Iron (II) Sulphide
            \item Iron (III) Sulphide
            \item None of these.
        \end{enumerate}
        \end{multicols}

    % ------- Question 1.iv ----------------------------------
    % ~   (iv) - c
    % ~       MLQB - 32/16
    \item Which of the following is a most reactive element of the group 17.

        \begin{multicols}{2}
        \begin{enumerate}[label=(\alph*)]
            \setlength\itemsep{0em}
            \item Oxygen
            \item Sodium
            \item Fluorine
            \item Magnesium
        \end{enumerate}
        \end{multicols}

    % ------- Question 1.v ----------------------------------
    % ~   (v) - d
    % ~       MLQB - 151/14
    \item An electrolyte which completely dissociates into ions is:

        \begin{multicols}{2}
        \begin{enumerate}[label=(\alph*)]
            \setlength\itemsep{0em}
            \item Alcohol
            \item Carbonic acid
            \item Sucrose
            \item Sodium hydroxide
        \end{enumerate}
        \end{multicols}

\end{enumerate}

% ======================================================================
% ~  Question 2
% ~  ----------
\par
\noindent
\textbf{Question 2}\\
\begin{enumerate}[label=(\roman*)]

    % ~   (2.i) 
    % ~     a) Period
    % ~     b) 89, 103
    % ~     c) low
    % ~     d) Acetic
    % ~     e) Solid
    % ~        
    \item \hfill [5]
        \begin{enumerate}[label=(\alph*)]
            \item Each \rule{1.5cm}{0.15mm} in the periodic table is comprised of elements 
                having the same number of shells.
            \item Actinides are the elements from atomic number \rule{1.5cm}{0.15mm} to 
                \rule{1.5cm}{0.15mm} and are radioactive.
            \item Melting and boiling points of covalent compounds are generally 
            \rule{1.5cm}{0.15mm}.\\ (high/low)
            \item Vinegar contains \rule{1.5cm}{0.15mm} acid.
            \item \rule{1.5cm}{0.15mm} lead bromide does not conduct electricity.
        \end{enumerate}

\end{enumerate}

