% ~ ........................................
% ~ Section A
% ~ ======================================== 
\begin{center}
   \large
   \textbf{SECTION A (20 marks)}\\
   \vspace{5mm}
   \normalsize
   \textit{(Attempt \textbf{all} questions from this \textbf{Section})}
\end{center}
\par
\noindent
% ~  Question 1
% ~  ----------
\textbf{Question 1}\\
Choose the correct answers to the questions from the given options. \hfill [7]\\
(Do not copy the questions, write the correct answers only.)
\par
\vspace{2mm}
\begin{enumerate}[label=(\roman*)]

    % ------- Question 1.i ----------------------------------
    % ~   (i) - (d) 
    % ~       [EC - Page 25 - Q 3]
    \item Naveen deposited \rupee~60 per month in a cumulative (recurring)
        deposit account for 4 years. The amount he gets at the time of maturity 
        if the rate of interest is 5\% per annum is:

        \begin{multicols}{2}
        \begin{enumerate}[label=(\alph*)]
            \item \rupee~3,240 
            \item \rupee~3,248.48
            \item \rupee~3,238.5
            \item \rupee~3,174
        \end{enumerate}
        \end{multicols}

    % ------- Question 1.ii ----------------------------------
    % ~   (ii) - (b) 
    % ~       [EC - Page 48 - Q 3]
    \newpage
    \item If $-4 < 2x + 6 < 2, x \in \mathbb{R}$, then $x$ lies in:

        \begin{multicols}{2}
        \begin{enumerate}[label=(\alph*)]
            \item $[-4, 6]$
            \item $(-5, -2)$
            \item $[-4, -2]$
            \item $[-5, 6]$
        \end{enumerate}
        \end{multicols}

    % ------- Question 1.iii ----------------------------------
    % ~   (iii) - (c)
    % ~       [EC - Page 65 - Q 3]
    \item The roots of $100x^2 - 20x + 1 =0$ is:

        \begin{multicols}{2}
        \begin{enumerate}[label=(\alph*)]
            \item $\dfrac{1}{20}$ and $\dfrac{1}{20}$
            \item $\dfrac{1}{10}$ and $\dfrac{1}{20}$
            \item $\dfrac{1}{10}$ and $\dfrac{1}{10}$
            \item None of these
        \end{enumerate}
        \end{multicols}

    % ------- Question 1.iv ----------------------------------
    % ~   (iv) - (b)
    % ~       [EC - Page 93 - Q 18]
    \item If $a$, $b$, $c$ and $d$ are in proportion then 
        $\sqrt{\dfrac{3a^2 + 8b^2}{3c^2 + 8d^2}}$ is equal to:

        \begin{multicols}{2}
        \begin{enumerate}[label=(\alph*)]
            \item $\dfrac{c}{a}$ 
            \item $\dfrac{b}{d}$ 
            \item $\dfrac{a}{b}$ 
            \item $\dfrac{c}{d}$ 
        \end{enumerate}
        \end{multicols}

    % ------- Question 1.v ----------------------------------
    % ~   (v) - (c)
    % ~       [EC - Page 124 - Q 22]
    \item The matrix $A = \begin{bmatrix*} 2\sin 30 \degree & \cos 0 \degree \\ 
          \cos 0 \degree & 2 \sin 30 \degree \end{bmatrix*}$ and 
          $B = \begin{bmatrix*} 1 \\ 2 \end{bmatrix*}$. If $AX = B$, then
          the order of matrix is:

        \begin{multicols}{2}
        \begin{enumerate}[label=(\alph*)]
            \item $1 \times 2$ 
            \item $2 \times 2$ 
            \item $2 \times 1$ 
            \item $1 \times 1$ 
        \end{enumerate}
        \end{multicols}

    % ------- Question 1.vi ----------------------------------
    % ~   (vi) - (c)
    % ~       [EC - Page 145 - Q 22]
    \item If the 8\textsuperscript{th} term of G.P. is 192 with a common
        ration of 2, then the 12\textsuperscript{th} term is:

        \begin{multicols}{2}
        \begin{enumerate}[label=(\alph*)]
            \item 1640 
            \item 2048
            \item 3072
            \item 31263
        \end{enumerate}
        \end{multicols}

    % ------- Question 1.vii ----------------------------------
    % ~   (vii) - (c)
    % ~       [EC - Page 301 - Q 1]
    \item The volume of a conial tent is 462 $m^3$ and the area 
        of base is 154 $m^2$. The height of the cone is:

        \begin{multicols}{2}
        \begin{enumerate}[label=(\alph*)]
            \item  15 m
            \item  12 m
            \item  9 m
            \item  24 m
        \end{enumerate}
        \end{multicols}

\end{enumerate}

% ======================================================================
% ~  Question 2
% ~  ----------
\newpage
\par
\noindent
\textbf{Question 2}\\
\begin{enumerate}[label=(\roman*)]

    % ~   (2.i) - Proof
    % ~       [EC - Page 151 - Q 48]
    \item If the $m\textsuperscript{th}$ term of an A.P. is 
        $\dfrac{1}{n}$ and the $n\textsuperscript{th}$ term is 
        $\dfrac{1}{m}$, show that the sum of $mn$ terms is 
        $\dfrac{1}{2}(mn + 1)$. \hfill [4]

    % ~   (2.ii) - x=5/8 
    % ~       [EC - Page 98 - Q 36]
    \item Using properties of proportion, solve for $x$. Given that 
        $x$ is positive: \hfill [4]
        \[
            \frac{2x + \sqrt{4x^2 - 1}}{2x - \sqrt{4x^2 - 1}} = 4
        \]

    % ~   (2.iii) -  1/25
    % ~       [EC - Page 71 - Q 32]
    \item The sum of a number and its positive square root is 
        $\dfrac{6}{25}$. What is the number? \hfill [5]

\end{enumerate}
