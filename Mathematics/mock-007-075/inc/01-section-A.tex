% ~ Section A
\begin{center}
   \large
   \textbf{SECTION A (20 marks)}\\
   \vspace{5mm}
   \normalsize
   \textit{(Attempt \textbf{all} questions from this \textbf{Section})}
\end{center}
\par
\noindent
% ~  Question 1
\textbf{Question 1}\\
Choose the correct answers to the questions from the given options. \hfill [7]\\
(Do not copy the questions, write the correct answers only.)
\par
\vspace{2mm}
\begin{enumerate}[label=(\roman*)]

    % ------- Question 1.i ----------------------------------
    % ~   (i) - (a) {1} 
    % ~       [MLQB - Page 61 - Q 3]
    \item If $2x-5 \leq 5x + 4 \leq 11$ and $x$ is a natural number $(\mathbb{N})$, then
        the solution set of $x$ is:

        \begin{multicols}{2}
        \begin{enumerate}[label=(\alph*)]
            \item \{ 1 \} 
            \item \{ -3, -2, -1, 0, 1 \} 
            \item \{ -2, -1, 0, 1 \} 
            \item \{ -2, -1, 0 \} 
        \end{enumerate}
        \end{multicols}

    % ------- Question 1.ii ----------------------------------
    % ~   (ii) - (d) 25:51
    % ~       [MLQB - Page 103 - Q 2]
    \item If $x:y = 5:3$, then the value of $(8x-5y):(6x+7y)$ is: 

        \begin{multicols}{2}
        \begin{enumerate}[label=(\alph*)]
            \item $24:27$
            \item $35:37$
            \item $25:9$
            \item $25:51$
        \end{enumerate}
        \end{multicols}

    % ------- Question 1.iii ----------------------------------
    % ~   (iii) - (c) (1-sin0)/(1+sin0)
    % ~       [MLQB - Page 354 - Q 14]
    \item Evaluate $(\sec \theta - \tan \theta)^2$: 

        \begin{multicols}{2}
        \begin{enumerate}[label=(\alph*)]
            \item $\dfrac{1 + \sin \theta}{\cos \theta}$ 
            \item $\dfrac{\sin \theta}{1 - \sin \theta}$ 
            \item $\dfrac{1 - \sin \theta}{1 + \sin \theta}$ 
            \item None of these
        \end{enumerate}
        \end{multicols}

    % ------- Question 1.iv ----------------------------------
    % ~   (iv) - (d) [[10,7][22, 17]]
    % ~       [MLQB - Page 139 - Q 7]
    \item If $A = \begin{bmatrix*} 2 & 3 \\ 4 & 2 \end{bmatrix*}$ and 
        $B = \begin{bmatrix*} 1 & 2 \\ 3 & 4 \end{bmatrix*}$, then the 
        matrix $BA$ is:

        \begin{multicols}{2}
        \begin{enumerate}[label=(\alph*)]
            \item $\begin{bmatrix*} 11 & 16 \\ 10 & 16 \end{bmatrix*}$ 
            \item $\begin{bmatrix*} 10 & 22 \\  7 & 17 \end{bmatrix*}$ 
            \item $\begin{bmatrix*} 11 & 10 \\ 16 & 16 \end{bmatrix*}$ 
            \item $\begin{bmatrix*} 10 &  7 \\ 22 & 17 \end{bmatrix*}$ 
        \end{enumerate}
        \end{multicols}

    % ------- Question 1.v ----------------------------------
    % ~   (v) - (c) (x+1),(2x+1)
    % ~       [MLQB - Page 129 - Q 15]
    \item The other factors of the polynomial $2x^3 - x^2 -5x -2$, if one
        of its factor is $(x-2)$ are:

        \begin{multicols}{2}
        \begin{enumerate}[label=(\alph*)]
            \item $(x+1),(2x-1)$ 
            \item $(x-1),(2x-1)$ 
            \item $(x+1),(2x+1)$ 
            \item $(x-1),(2x+1)$ 
        \end{enumerate}
        \end{multicols}

    % ------- Question 1.vi ----------------------------------
    % ~   (vi) - (d) k < 3
    % ~       [MLQB - Page 75 - Q 6]
    \item If the equation $3x^2 - 6x + k = 0$ has real and distinct roots, then the 
        value of $k$ is:

        \begin{multicols}{2}
        \begin{enumerate}[label=(\alph*)]
            \item $k \leq 3$ 
            \item $k = 3$ 
            \item $k > 3$ 
            \item $k < 3$ 
        \end{enumerate}
        \end{multicols}

    % ------- Question 1.vii ----------------------------------
    % ~   (vii) - (b) Rs. 16937.50
    % ~       [MLQB - Page 41 - Q 6]
    \item Simran had a recurring deposit account in a bank and deposited 
        \rupee~500 per month for $2\dfrac12$ years. If the rate of interest was
        10\% p.a., then the matured value of this account is:

        \begin{multicols}{2}
        \begin{enumerate}[label=(\alph*)]
            \item \rupee~16,397.50 
            \item \rupee~16,937.50 
            \item \rupee~16,793.50 
            \item \rupee~16,973.50 
        \end{enumerate}
        \end{multicols}

\end{enumerate}

% ======================================================================
% ~  Question 2
\newpage
\par
\noindent
\textbf{Question 2}\\
\begin{enumerate}[label=(\roman*)]

    % ~   (2.i) - k = -1 or 2/3
    % ~       [MLQB - Page 82 - Q 43]
    \item Find the value of $k$ for which the following 
        equation has equal roots: \hfill [4]
        \[
            x^2 + 4kx + (k^2 - k + 2) = 0
        \]

    % ~   (2.ii) - 7/6 
    % ~       [MLQB - Page 358 - Q 31]
    \item If $5 \tan \theta = 4$, find the value of: \hfill [4]
        \[
            \frac{5 \sin \theta + 3 \cos \theta}{5 \sin \theta + 2 \cos \theta}
        \]

    % ~   (2.iii) - x = +-2 
    % ~       [MLQB - Page 110 - Q 35]
    \item Using the properties of proportion, solve for $x$, given \hfill [5]
        \[
            \frac{x^4 + 1}{2x^2} = \frac{17}{8}
        \]

\end{enumerate}
