% ~ ........................................
% ~ Section B
% ~ ======================================== 
\newpage
\begin{center}
   \large
   \textbf{SECTION B (20 marks)}\\
   \vspace{5mm}
   \normalsize
   \textit{(Attempt \textbf{any two} questions from this \textbf{Section})}
\end{center}
\par

% ======================================================================
% ~  Question 3
% ~  ----------
\noindent
\textbf{Question 3}
\begin{enumerate}[label=(\roman*)]

    % ~   (3.i) - {x:x>-6/7, in in R} 
    % ~       [MLQB - Page 67 - Q 34.vii]
    \item Solve the following ineuqation and represent the solution on 
        a number line: \hfill [3]
        \[
            \frac{3x}{2} + \frac{1}{4} > \frac{5x}{8} - \frac12
        \]

    % ~   (3.ii) - 34,50,000 
    % ~       [MLQB - Page 164 - Q 49]
    \item The income of a person is \rupee~3,00,000, in the first year 
        and he receives an increase of \rupee~10,000 to his income per year
        for the next 9 years. Determine the total amount he received in 
        10 years. \hfill [3]

    % ~   (3.iii) - 61% 
    % ~       [MLQB - Page 397 - Q 35]
    \item In X standard, there are three sections A, B and C with 25, 40 
        and 35 students, respectively. The average marks of section A is 
        70\%, section B is 65\% and of section C is 50\%. 

        Find the avrage marks of the entire X standard.\hfill [4]

\end{enumerate}

% ======================================================================
% ~  Question 4
% ~  ----------
\noindent
\textbf{Question 4}
\begin{enumerate}[label=(\roman*)]

    % ~   (4.i) - 18 
    % ~       [MLQB - Page 164 - Q 48]
    \item Using A.P. determine how many two digit numbers are divisible by 5. \hfill [3]

    % ~   (4.ii) - -4.5 <= x <= 1.5 
    % ~       [MLQB - Page 73 - Q 3]
\item Solve $\abs{ 2x + 3 } \geq -6, x \in \mathbb{R}$ and represet 
    the solution on a number line. \hfill [3]

    % ~   (4.iii) - x=2, y=10 
    % ~       [MLQB - Page 145 - Q 36]
    \item Determine the value of $x$ and $y$ if: \hfill [4]
        \[
            2 \begin{bmatrix*}  x &  7 \\  9 & y-5 \end{bmatrix*} + 
              \begin{bmatrix*}  6 & -7 \\  4 &   5 \end{bmatrix*} =
              \begin{bmatrix*} 10 &  7 \\ 22 &  15 \end{bmatrix*}
        \]

\end{enumerate}

% ======================================================================
% ~  Question 5
% ~  ----------
\newpage
\noindent
\textbf{Question 5}
\begin{enumerate}[label=(\roman*)]

    % ~   (5.i) - Proof 
    % ~       [MLQB - Page 367 - Q 69]
    \item Prove that: \hfill [4]
        \[
            \frac{\sin \theta}{\sec \theta + \tan \theta - 1} + 
            \frac{\cos \theta}{\mathrm{cosec} \ \theta + \cot \theta -1 } = 1
        \]

    % ~   (5.ii) - 53.424 and 122.708 
    % ~       [MLQB - Page 374 - Q 95]
    \item The horizontal distance between two towers is 120 m. The angle 
        of elevation of the top and angle of depression of the bottom 
        of the first tower as obseved from the second tower is 30\degree 
        and 24\degree respectively. Find the height of two towers correct 
        to three decimal places. Use $\tan 25\degree = 0.4452$ \hfill [6]
        \img{5cm}{5.ii.png}

\end{enumerate}


