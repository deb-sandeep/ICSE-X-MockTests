% ~ ........................................
% ~ Section A
% ~ ======================================== 
\begin{center}
   \large
   \textbf{SECTION A (20 marks)}\\
   \vspace{5mm}
   \normalsize
   \textit{(Attempt \textbf{all} questions from this \textbf{Section})}
\end{center}
\par
\noindent
% ~  Question 1
% ~  ----------
\textbf{Question 1}\\
Choose the correct answers to the questions from the given options. \hfill [7]\\
(Do not copy the questions, write the correct answers only.)
\par
\vspace{2mm}
\begin{enumerate}[label=(\roman*)]

    % ------- Question 1.i ----------------------------------
    % ~   (i) - (c) 
    % ~       [MLQB - Page 388 - Q1]
    \item Determine the mean of the following frequency distribution:
        \begin{table}[h]
        \centering
        \renewcommand{\arraystretch}{1.3}
        \begin{tabularx}{0.5\textwidth}{| p {2.75 cm} | X | X | X | X | X | }
            \hline
            Group & 10-15 & 15-20 & 20-25 \\
            \hline
            Frequency & 3 & 10  & 2 \\
            \hline
        \end{tabularx}
        \end{table}
        

        \begin{multicols}{2}
        \begin{enumerate}[label=(\alph*)]
            \item 15.1
            \item 15.17
            \item 17.17
            \item 17.71
        \end{enumerate}
        \end{multicols}

    % ------- Question 1.ii ----------------------------------
    % ~   (ii) - (c) 
    % ~       [MLQB - Page 158 - Q 6]
    \newpage
    \item If there are 15 terms in an A.P. whose first term is 
        $\sqrt{2}$ and common difference is $2\sqrt{2}$, then the
        last term is:

        \begin{multicols}{2}
        \begin{enumerate}[label=(\alph*)]
            \item $31\sqrt{2}$ 
            \item $30\sqrt{2}$ 
            \item $29\sqrt{2}$ 
            \item $28\sqrt{2}$ 
        \end{enumerate}
        \end{multicols}

    % ------- Question 1.iii ----------------------------------
    % ~   (iii) - (d)
    % ~       [MLQB - Page 141 - Q 19]
    \item If $A = \begin{bmatrix*} 3 & 1 \\ 7 & 5 \end{bmatrix*}$ and 
        $A^2 + xI = yA$, then the value of $x$ is:

        \begin{multicols}{2}
        \begin{enumerate}[label=(\alph*)]
            \item -8 
            \item -4 
            \item 4 
            \item 8 
        \end{enumerate}
        \end{multicols}

    % ------- Question 1.iv ----------------------------------
    % ~   (iv) - (b)
    % ~       [MLQB - Page 126 - Q 1]
    \item The remainder when $f(x) = x^2 - 4x + 2$ is divided by $2x+1$, is: 

        \begin{multicols}{2}
        \begin{enumerate}[label=(\alph*)]
            \item $\frac{1}{4}$ 
            \item $\frac{17}{4}$ 
            \item -10
            \item 22
        \end{enumerate}
        \end{multicols}

    % ------- Question 1.v ----------------------------------
    % ~   (v) - (a)
    % ~       [MLQB - Page 77 - Q 20]
    \item If length of hypotenuse of a right-angled triangle exceeds the length 
        of one side by 2 cm and exceeds twice the length of other side by 1 cm,
        then the length of hypotenuse of the triangle is:

        \begin{multicols}{2}
        \begin{enumerate}[label=(\alph*)]
            \item 17 cm
            \item 1 cm
            \item 15 cm
            \item 22 cm
        \end{enumerate}
        \end{multicols}

    % ------- Question 1.vi ----------------------------------
    % ~   (vi) - (a)
    % ~       [MLQB - Page 353 - Q 4]
    \item Evaluate: $\sin^2 \theta + \tan^2 \theta + \cos^2 \theta$:

        \begin{multicols}{2}
        \begin{enumerate}[label=(\alph*)]
            \item $\sec^2 \theta$
            \item 1
            \item 0
            \item $2 \sin \theta$
        \end{enumerate}
        \end{multicols}

    % ------- Question 1.vii ----------------------------------
    % ~   (vii) - (b)
    % ~       [MLQB - Page 159 - Q 13]
    \item Tenth term from the end of A.P. 18, 16, 14, \dots, -10 is: 

        \begin{multicols}{2}
        \begin{enumerate}[label=(\alph*)]
            \item 0
            \item 8
            \item -2
            \item 10
        \end{enumerate}
        \end{multicols}

\end{enumerate}

% ======================================================================
% ~  Question 2
% ~  ----------
\newpage
\par
\noindent
\textbf{Question 2}\\
\begin{enumerate}[label=(\roman*)]

    % ~   (2.i) - Proof
    % ~       [MLQB - Page 360 - Q 42]
    \item Prove that \hfill [4]
        \[
            \sin^2 \theta - \cos^4 \theta = 1 - 2 \cos^2 \theta
        \]

    % ~   (2.ii) - P=7
    % ~       [MLQB - Page 400 - Q 50]
    \item The mean of the following distribution is 6. Find the 
        value of P. \hfill [4]
        \begin{table}[h]
        \centering
        \renewcommand{\arraystretch}{1.3}
        \begin{tabularx}{0.6\textwidth}{| p {2.75 cm} | X | X | X | X | X | }
            \hline
            Variate (x) & 2 & 4 & 6 & 10 & P+5 \\
            \hline
            Frequency (f) & 3 & 2 & 3 & 1 & 2 \\
            \hline
        \end{tabularx}
        \end{table}
        

    % ~   (2.iii) - 25.36m
    % ~       [MLQB - Page 373 - Q 92]
    \item From the top of a tower 60m high, the angles of depression
        of the top and bottom of pole are observed to be 45\degree and 
        60\degree respectively. Find the height of the pole. \hfill [5]

        \img{6cm}{2_3.png}

\end{enumerate}
