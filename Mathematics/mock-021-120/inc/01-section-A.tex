% ~ ........................................
% ~ Section A
% ~ ======================================== 
\begin{center}
   \large
   \textbf{SECTION A (34 marks)}\\
   \vspace{5mm}
   \normalsize
   \textit{(Attempt \textbf{all} questions from this \textbf{Section})}
\end{center}
\par
\noindent
% ~  Question 1
% ~  ----------
\textbf{Question 1}\\
Choose the correct answers to the questions from the given options. \hfill [14]\\
(Do not copy the questions, write the correct answers only.)
\par
\vspace{2mm}
\begin{enumerate}[label=(\roman*)]

    % ------- Question 1.i ------------------------------------
    % ~   (i) - b
    % ~       MLQB - 64/20
    \item Which of the following linear inequations has the solution set 
        $\{-1, 0, 1, 2, 3 \}$?

        \begin{enumerate}[label=(\alph*)]
            \item $\dfrac23 + \dfrac13 (x+1) > 0, x \in \mathbb{I}$
            \item $2(x-2) < 3x-2 < 10 , x \in \mathbb{I}$
            \item $5x + 7 > 27, x \in \mathbb{I}$
            \item $3x + 12 < 0, x \in \mathbb{I}$
        \end{enumerate}

    % ------- Question 1.ii ----------------------------------
    % ~   (ii) - b
    % ~       MLQB - 210/4
    \newpage
    \item Point $P(a,b)$ is reflected in the $y$-axis to $P^{\prime}(3, -2)$. 
        The values of $a$ and $b$ are:

        \begin{multicols}{2}
        \begin{enumerate}[label=(\alph*)]
            \item $a=-5, b=2$
            \item $a=-3, b=-2$
            \item $a=5, b=-2$
            \item $a=2, b=5$
        \end{enumerate}
        \end{multicols}

    % ------- Question 1.iii ----------------------------------
    % ~   (iii) - c
    % ~       MLQB - 52/8
    \item Shahina invests \rupee~9620 on \rupee~100 shares at \rupee~80 if 
        the company pays her 18\% dividend then her total dividend is:

        \begin{multicols}{2}
        \begin{enumerate}[label=(\alph*)]
            \item 2100
            \item 2200
            \item 2160
            \item 2106
        \end{enumerate}
        \end{multicols}

    % ------- Question 1.iv ----------------------------------
    % ~   (iv) - c
    % ~       MLQB - 104/5
    \item The ratio between 3.6 m and 75 cm is:

        \begin{multicols}{2}
        \begin{enumerate}[label=(\alph*)]
            \item $18 : 275$
            \item $6 : 125$
            \item $24 : 5$
            \item $4 : 5$
        \end{enumerate}
        \end{multicols}

    % ------- Question 1.v ----------------------------------
    % ~   (v) - c
    % ~       MLQB - 268/3
    \item In the given figure, O is the center of the circle. 
        $\angle OQP = 25\degree$ and $\angle ORP = 55\degree$. Find the 
        $\angle QOR$.

        \img{0.3\textwidth}{1-v.png}

        \begin{multicols}{2}
        \begin{enumerate}[label=(\alph*)]
            \item 120\degree
            \item 140\degree
            \item 160\degree
            \item 110\degree
        \end{enumerate}
        \end{multicols}

    % ------- Question 1.vi ----------------------------------
    % ~   (vi) - b
    % ~       MLQB - 356/22
    \item Evaluate: $(\sin \theta + \cos \theta + 1)^2$, when $\sin \theta \cos \theta = 0$:

        \begin{multicols}{2}
        \begin{enumerate}[label=(\alph*)]
            \item $2 \sin \theta \cos \theta$
            \item $2 + 2(\sin \theta + \cos \theta)$
            \item $2 \cos \theta + \sin \theta$
            \item None of these
        \end{enumerate}
        \end{multicols}

    % ------- Question 1.vii ----------------------------------
    % ~   (vii) - b
    % ~       MLQB - 30/11
    \newpage
    \item If a customer bought two items worth \rupee~450 and \rupee~200
        at 20\% and 10\% discount respectively, then the amount of bill
        for the intra-state transaction if the rate of GST is 18\%, is:

        \begin{multicols}{2}
        \begin{enumerate}[label=(\alph*)]
            \item \rupee~622.20
            \item \rupee~637.20
            \item \rupee~657
            \item \rupee~700
        \end{enumerate}
        \end{multicols}

    % ------- Question 1.viii ---------------------------------
    % ~   (viii) - b
    % ~       MLQB - 75/7
    \item One root of the quadratic equation $3x^2 - 4x - 4 = 0$ is:

        \begin{multicols}{2}
        \begin{enumerate}[label=(\alph*)]
            \item $\dfrac32$
            \item 2
            \item $\dfrac23$
            \item 6
        \end{enumerate}
        \end{multicols}

    % ------- Question 1.ix -----------------------------------
    % ~   (ix) - a
    % ~       MLQB - 159/17
    \item The value of $n$, for which the $n^{th}$ term of A.P. 
        63, 65, 67, \dots is equal to the $n^{th}$ term of A.P. 
        3, 10, 17, \dots are equal to each other is:

        \begin{multicols}{2}
        \begin{enumerate}[label=(\alph*)]
            \item 13
            \item 12
            \item 15
            \item 17
        \end{enumerate}
        \end{multicols}

    % ------- Question 1.x ------------------------------------
    % ~   (x) - b
    % ~       MLQB - 42/11
    \item Mr. Nair gets \rupee~6,455 at the end of one year at the rate of 
        14\% per annum in a recurring deposit amount. The monthly instalment is:

        \begin{multicols}{2}
        \begin{enumerate}[label=(\alph*)]
            \item \rupee~400
            \item \rupee~500
            \item \rupee~600
            \item \rupee~700
        \end{enumerate}
        \end{multicols}

    % ------- Question 1.xi -----------------------------------
    % ~   (xi) - c
    % ~       MLQB - 128/9
    \item If $x-2$ is a factor of $x^3 + 2x^2 - kx + 10$, then the value of $k$ is:

        \begin{multicols}{2}
        \begin{enumerate}[label=(\alph*)]
            \item 3
            \item 8
            \item 13
            \item 26
        \end{enumerate}
        \end{multicols}

    % ------- Question 1.xii ----------------------------------
    % ~   (xii) - a
    % ~       MLQB - 140/13
    \item If $A = \begin{bmatrix*} -2 & 3 \\ 4 & 5 \end{bmatrix*}$ and 
        $B = \begin{bmatrix*} 5 & 2 \\ -7 & 3 \end{bmatrix*}$, then the matrix $C$,
        such that $A + B - C = 0$ is:

        \begin{multicols}{2}
        \begin{enumerate}[label=(\alph*)]
            \item $\begin{bmatrix*} 3 & 5 \\ -3 & 8 \end{bmatrix*}$
            \item $\begin{bmatrix*} -7 & 1 \\ 11 & 2  \end{bmatrix*}$
            \item $\begin{bmatrix*} 7 & -1 \\ -11 & -2 \end{bmatrix*}$
            \item $\begin{bmatrix*} -3 & -5 \\ 3 & -8 \end{bmatrix*}$
        \end{enumerate}
        \end{multicols}

    % ------- Question 1.xiii ----------------------------------
    % ~   (xiii) - c
    % ~       MLQB - 175/3
    \newpage
    \item The 11\textsuperscript{th} term of the G.P. 
        $\dfrac18, -\dfrac14, 2, -1, \dots$ is:

        \begin{multicols}{2}
        \begin{enumerate}[label=(\alph*)]
            \item 64
            \item -64
            \item 128
            \item -128
        \end{enumerate}
        \end{multicols}

    % ------- Question 1.xiv ----------------------------------
    % ~   (xiv) - a
    % ~       MLQB - 53/18
    \item Anita buys 400, twenty rupees shares at a discount of 20\% and 
        receives a return of 12\% on her money. Calculated the amount invested
        by her.

        \begin{multicols}{2}
        \begin{enumerate}[label=(\alph*)]
            \item \rupee~6400
            \item \rupee~6401
            \item \rupee~6500
            \item \rupee~6600
        \end{enumerate}
        \end{multicols}

\end{enumerate}

% ======================================================================
% ~  Question 2
% ~  ----------
\par
\noindent
\textbf{Question 2}\\
\begin{enumerate}[label=(\roman*)]

    % ~   (2.i) - (x+1)(x-2)(x+2)
    % ~       MLQB - 135/45
    \item Use the factor theorem to factorize completely : 
        $x^3 + x^2 - 4x - 4$. \hfill [3]

    % ~   (2.ii) - v1:v2 = 1:2
    % ~       MLQB - 328/35
    \item Two cylinders have bases of same size. The diameter of each is 14 cm. 
        One of the cylinders is 10 cm high and the other is 20 cm high. Find 
        the ratio between their volumes. \hfill [3]

    % ~   (2.iii) - Total camels = 36 
    % ~       MLQB - 92/81
    \item One-fourth of a herd of camels was seen in the forest. Twice the 
        square root of the herd had gone to mountains and the remaining 15 camels 
        were seen on the bank of a river. Find the total number of camels. \hfill [4]

\end{enumerate}

% ======================================================================
% ~  Question 3
% ~  ----------
\newpage
\noindent
\textbf{Question 3}\\
\begin{enumerate}[label=(\roman*)]

    % ~   (3.i) - 1 + cosA sinA
    % ~       MLQB - 355/19
    \item Evaluate: \hfill [3]
        \[
            \frac{\cos^2 \theta}{1 - \tan \theta} + 
            \frac{\sin^3 \theta}{\sin \theta - \cos \theta}
        \]

    % ~   (3.ii) - ABC = 36 degree, BAD = 36 degree, ABD = 54 degree
    % ~       MLQB - 290/83
    \item In the figure given below, O is the center of the circle and 
        AB is a diameter. \hfill [3]

        If $AC = BD$ and $\angle AOC = 72 \degree$, Find:
        \begin{multicols}{3}
        \begin{enumerate}
            \setlength\itemsep{0pt}
            \item $\angle ABC$
            \item $\angle BAD$
            \item $\angle ABD$
        \end{enumerate}
        \end{multicols}

        \img{0.4\textwidth}{3-ii.png}

    % ~   (3.iii) - i) 500, ii) 3000 and iii) 13%
    % ~       MLQB - 55/30
    \item A man invests \rupee~22,500 in \rupee~50 shares available at 
        10\% discount. If the dividend paid by the company is 12\%, calculate: \hfill [4]
        \begin{enumerate}
            \setlength\itemsep{0pt}
            \item The number of shares purchased.
            \item The annual dividend received.
            \item The rate of return he gets on his investment. Give your answer 
                correct to the nearest whole number.
        \end{enumerate}


\end{enumerate}

