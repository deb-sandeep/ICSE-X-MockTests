% ~ Section A
\begin{center}
   \large
   \textbf{SECTION A (20 marks)}\\
   \vspace{5mm}
   \normalsize
   \textit{(Attempt \textbf{all} questions from this \textbf{Section})}
\end{center}
\par
\noindent
% ~  Question 1
\textbf{Question 1}\\
Choose the correct answers to the questions from the given options. \hfill [7]\\
(Do not copy the questions, write the correct answers only.)
\par
\vspace{2mm}
\begin{enumerate}[label=(\roman*)]

    % ------- Question 1.i ----------------------------------
    % ~   (i) - (c) Rs. 9270 
    % ~       [MLQB - Page 42 - Q10]
    \item Kishan deposited \rupee~360 per month in a cumulative time deposit 
        account for 2 years. If the rate of interest is 7\% per annum,
        then the amount he gets at the time of maturity is:

        \begin{multicols}{2}
        \begin{enumerate}[label=(\alph*)]
            \item \rupee~2790 
            \item \rupee~9720
            \item \rupee~9270
            \item \rupee~7290 
        \end{enumerate}
        \end{multicols}

    % ------- Question 1.ii ----------------------------------
    % ~   (ii) - (c) 2:1 
    % ~       [MLQB - Page 104 - Q8]
    \newpage
    \item If $x^2 + 4y^2 = 4xy$, then $x:y$ is: 

        \begin{multicols}{2}
        \begin{enumerate}[label=(\alph*)]
            \item 1:4 
            \item 4:1 
            \item 2:1 
            \item 1:2 
        \end{enumerate}
        \end{multicols}

    % ------- Question 1.iii ----------------------------------
    % ~   (iii) - (a) 5
    % ~       [MLQB - Page 127 - Q6]
    \item If $(x-2)$ is a factor of $2x^3 - x^2 - px - 2$, then the value of 
        $p$ is:

        \begin{multicols}{2}
        \begin{enumerate}[label=(\alph*)]
            \item 5
            \item 4
            \item 10
            \item 8
        \end{enumerate}
        \end{multicols}

    % ------- Question 1.iv ----------------------------------
    % ~   (iv) - (c) 3
    % ~       [MLQB - Page 63 - Q13]
    \item If $25 - 4x \leq 16, x \in \mathbb{N}$, then the smallest value of 
        x is:

        \begin{multicols}{2}
        \begin{enumerate}[label=(\alph*)]
            \item $\dfrac94$
            \item 2
            \item 3
            \item None of these
        \end{enumerate}
        \end{multicols}

    % ------- Question 1.v ----------------------------------
    % ~   (v) - (b) 6.54, 0.46 
    % ~       [MLQB - Page 75 - Q9]
    \item The roots of the quadratic equation $x^2 - 7x + 3 = 0$, are: 

        \begin{multicols}{2}
        \begin{enumerate}[label=(\alph*)]
            \item -6.54, -0.46 
            \item 6.54, 0.46 
            \item 6.54, -0.46 
            \item -6.54, 0.46 
        \end{enumerate}
        \end{multicols}

    % ------- Question 1.vi ----------------------------------
    % ~   (vi) - (a) -1/9
    % ~       [MLQB - Page 353 - Q5]
    \item If $\sin \theta = \dfrac{15}{17}$, find the value of 
        $\dfrac{3 \cos \theta - 2 \sin \theta}{3 \cos \theta + 2 \sin \theta}$

        \begin{multicols}{2}
        \begin{enumerate}[label=(\alph*)]
            \item $\dfrac{-1}{9}$ 
            \item $\dfrac{1}{19}$ 
            \item $\dfrac{1}{7}$ 
            \item $\dfrac{-1}{17}$ 
        \end{enumerate}
        \end{multicols}

    % ------- Question 1.vii ----------------------------------
    % ~   (vii) - (c) 1, 5/2
    % ~       [MLQB - Page 139 - Q5]
    \item If $x \begin{bmatrix*}[r] -1 \\ 2 \end{bmatrix*} +
              4 \begin{bmatrix*}[r] 2 \\ -y \end{bmatrix*} =
              \begin{bmatrix*}[r] 7 \\ -8 \end{bmatrix*}$, then
        the respective values of $x$ and $y$, are:

        \begin{multicols}{2}
        \begin{enumerate}[label=(\alph*)]
            \item $-1, \dfrac32$ 
            \item $15, \dfrac{19}{2}$ 
            \item $1, \dfrac52$ 
            \item $-1, \dfrac52$ 
        \end{enumerate}
        \end{multicols}

\end{enumerate}

% ======================================================================
% ~  Question 2
\newpage
\par
\noindent
\textbf{Question 2}\\
\begin{enumerate}[label=(\roman*)]

    % ~   (2.i) - [[1,0][0,1]]
    % ~       [MLQB - Page 144 - Q 34]
    \item Simplify \hfill [4]
        \[ 
            \sin A \begin{bmatrix*}[r] \sin A & -\cos A \\ \cos A & \sin A \end{bmatrix*} +
            \cos A \begin{bmatrix*}[r] \cos A & \sin A \\ -\sin A & \cos A \end{bmatrix*} 
        \]

    % ~   (2.ii) - Proof 
    % ~       [MLQB - Page 361 - Q 51]
    \item Prove that: \hfill [4]
        \[
            \frac{1 + \sin \theta}{1 - \sin \theta} = 1 + 2\frac{\tan \theta}{\cos \theta} + 2 \tan^2 \theta
        \]

        % ~   (2.iii) - {-3, -2, -1, 0, 1, 2, 3, 4} 
    % ~       [MLQB - Page 69 - Q41]
    \item Solve the following equation and represent the solution set on the
        number line \hfill [4]
        \[
            -3 + x \leq \frac{8x}{3} + 2 \leq \frac{14}{3} + 2x, x \in \mathbb{I}
        \]

\end{enumerate}
