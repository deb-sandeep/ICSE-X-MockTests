% ~ Section B
\newpage
\begin{center}
   \large
   \textbf{SECTION B (20 marks)}\\
   \vspace{5mm}
   \normalsize
   \textit{(Attempt \textbf{any two} questions from this \textbf{Section})}
\end{center}
\par

% ======================================================================
% ~  Question 3
\noindent
\textbf{Question 3}
\begin{enumerate}[label=(\roman*)]

    % ~   (3.i) - Rs. 8421
    % ~       [MLQB - Page 48 - Q 36]
    \item Kiran deposited \rupee~200 per month for 36 months in a bank's
        recurring depoist account. If the bank pays interest at the rate
        of 11\% per annum, find the amount she gets on maturity. \hfill [3]

    % ~   (3.ii) - Roots are -8, 1
    % ~       [MLQB - Page 83 - Q 47]
    \item Solve $x^\frac23 + x^\frac13 -2 = 0$ \hfill [3]

    % ~   (3.iii) - Original number of children = 50 
    % ~       [MLQB - Page 93 - Q 84]
    \item \rupee~7500 were divided equally amongst a certain number of
        children. Had there been 20 less children, each would have received
        \rupee~100 more. Find the original number of children. \hfill [4]

\end{enumerate}

% ======================================================================
% ~  Question 4
\noindent
\textbf{Question 4}
\begin{enumerate}[label=(\roman*)]

    % ~   (4.i) - K = 5 or -1
    % ~       [MLQB - Page 111 - Q 40]
    \item The following numbers, $K+3$, $K+2$, $3K - 7$ and $2K-3$ are 
        in proportion. \hfill [3]
        Find K.

    % ~   (4.ii) - k=7
    % ~       [MLQB - Page 135 - Q 47]
    \item What must be added to the polynomial $2x^3 - 3x^2 -8x$, so that
        it leaves a remainder $10$ when divided by $2x+1$? \hfill [3]

    % ~   (4.iii) - x = 3 and y = 2 
    % ~       [MLQB - Page 147 - Q 48]
    \item Determind $x$ and $y$, if: \hfill [4]
        \[
            \begin{bmatrix*}[r] 3 & -2 \\ -1 & 4 \end{bmatrix*}
            \begin{bmatrix*}[r] 2x \\ 1 \end{bmatrix*} + 
            2 \begin{bmatrix*}[r] -4 \\ 5 \end{bmatrix*} = 
            4 \begin{bmatrix*}[r] 2 \\ y \end{bmatrix*}
        \]

\end{enumerate}

% ======================================================================
% ~  Question 5
\noindent
\textbf{Question 5}
\begin{enumerate}[label=(\roman*)]

    % ~   (5.i) - Proof
    % ~       [MLQB - Page 363 - Q 57]
    \item Prove that \hfill [4]
        \[
            \left( \frac{1 - \cos^2 \theta}{\cos \theta} \right)
            \left( \frac{1 - \sin^2 \theta}{\sin \theta} \right) =
            \frac{1}{\tan \theta + \cot \theta}
        \]

        % ~   (5.ii) - {1, 2, 3}
    % ~       [MLQB - Page 1 - Q9]
    \item Solve and represent the solution on a number line.\hfill [6]
        \[
            -2 \leq \frac12 - \frac{2x}{3} \leq 1\frac56, \ x \in \mathbb{N}
        \]

\end{enumerate}


