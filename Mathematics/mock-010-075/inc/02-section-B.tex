% ~ ........................................
% ~ Section B
% ~ ======================================== 
\newpage
\begin{center}
   \large
   \textbf{SECTION B (20 marks)}\\
   \vspace{5mm}
   \normalsize
   \textit{(Attempt \textbf{any two} questions from this \textbf{Section})}
\end{center}
\par

% ======================================================================
% ~  Question 3
% ~  ----------
\noindent
\textbf{Question 3}
\begin{enumerate}[label=(\roman*)]

    % ~   (3.i) - Rs 1600 
    % ~       [MLQB - Page 47 - Q 32]
    \item Zafarullah has a recurring deposit account in a bank for 
        $3\frac12$ years at 9.5\% p.a. If he gets \rupee~78,638 
        at the time of maturity, find the monthly installment. \hfill [3]

    % ~   (3.ii) - {-3, -2, -1,..., 3, 4} 
    % ~       [Balal 2]
    \item Solve and represent the solution on a number line. \hfill [3]
        \[
            -3 + x \leq \frac{8x}{3} + 2 \leq \frac{14}{3} + 2x, x \in \mathbb{I}
        \]

    % ~   (3.iii) - 17 m
    % ~       [MLQB - Page 372 - Q 89]
    \item The angle of elevation from a point P of the top of 
        a towner QR, 50 m high is 60\degree \ and that of the tower
        PT from a point Q is 30\degree. Find the height of the 
        tower PT, correct to the nearest metre. \hfill [4]
       \img{4cm}{372.89.png}

\end{enumerate}

% ======================================================================
% ~  Question 4
% ~  ----------
\noindent
\textbf{Question 4}
\begin{enumerate}[label=(\roman*)]

    % ~   (4.i) - n=26 
    % ~       [MLQB - Page 164 - Q 44]
    \item Determine which term of the sequence 4, 9, 14, 19, \dots is 129. \hfill [3]

    % ~   (4.ii) - a=3, b=15 
    % ~       [MLQB - Page 134 - Q 42]
    \item Find the value of $a$ and $b$ so that the polynomial
        $x^3 - ax^2 - 13x + b$ has $(x-1)(x+3)$ as factor. \hfill [3]

    % ~   (4.iii) - Proof 
    % ~       [MLQB - Page 367 - Q 65]
    \item Prove that: \hfill [4]
        \[
            \sqrt{\frac{1 + \cos \theta}{1 - \cos \theta}} = 
            \frac{\tan \theta + \sin \theta}{\tan \theta \sin \theta}
        \]

\end{enumerate}

% ======================================================================
% ~  Question 5
% ~  ----------
\newpage
\noindent
\textbf{Question 5}
\begin{enumerate}[label=(\roman*)]

    % ~   (5.i) - [[-23, 3][17, 6]] 
    % ~       [MLQB - Page 150 - Q 59]
    \item Let $A = \begin{bmatrix*} 2 & 1 \\ 0 & -2 \end{bmatrix*}$,
          $B = \begin{bmatrix*} 4 & 1 \\ -3 & -2 \end{bmatrix*}$ and 
          $C = \begin{bmatrix*} -3 & 2 \\ -1 & 4 \end{bmatrix*}$. Determine: \hfill [4]
          \[
            A^2 + AC - 5B
          \]

    % ~   (5.ii) - Median = 150 cm, Lower quartile = 146 cm, #tall boys = 9 
    % ~       [MLQB - Page 410 - Q 75]
    \item Use graph paper for the following question. \hfill [6]

        A survey regarding height (in cm) of 60 boys belonging to Class 10 of
        a school was conducted. The following data was recorded:

        \begin{table}[h]
        \centering
        \renewcommand{\arraystretch}{1.3}
        \begin{tabularx}{0.5\textwidth}{| p {3 cm} | X | }
            \hline
            \rowcolor{lightgray!30} Height (in cm) & No. of boys \\
            \hline
            135 – 140 & 4 \\
            \hline
            140 – 145 & 8 \\
            \hline
            145 – 150 & 20 \\
            \hline
            150 – 155 & 14 \\
            \hline
            155 – 160 & 7 \\
            \hline
            160 – 165 & 6 \\
            \hline
            165 – 170 & 1 \\
            \hline
        \end{tabularx}
        \end{table}

        Taking 2 cm = height of 10 cm along one axis and 2 cm = 10 boys
        along the other axis draw and ogive of the above distribution. 
        Use the graph to estimate the following:

        \begin{itemize}
            \item The Median
            \item Lower quartile
            \item If above 158 cm is considered as the tall boys of the 
                  class. Find the number of boys in the class who are tall.
        \end{itemize}

\end{enumerate}


