% ~ ........................................
% ~ Section A
% ~ ======================================== 
\begin{center}
   \large
   \textbf{SECTION A (20 marks)}\\
   \vspace{5mm}
   \normalsize
   \textit{(Attempt \textbf{all} questions from this \textbf{Section})}
\end{center}
\par
\noindent
% ~  Question 1
% ~  ----------
\textbf{Question 1}\\
Choose the correct answers to the questions from the given options. \hfill [7]\\
(Do not copy the questions, write the correct answers only.)
\par
\vspace{2mm}
\begin{enumerate}[label=(\roman*)]

    % ------- Question 1.i ----------------------------------
    % ~   (i) - (a) 
    % ~       [MLQB - Page 354 - Q 10]
    \item Find the value of $\dfrac{\sin \theta - 2 \sin^3 \theta}{ 2 \cos^3 \theta - \cos \theta}$: 

        \begin{multicols}{2}
        \begin{enumerate}[label=(\alph*)]
            \item $\tan \theta$ 
            \item $\cot \theta$ 
            \item $\sin \theta$ 
            \item $\cos \theta$ 
        \end{enumerate}
        \end{multicols}

    % ------- Question 1.ii ----------------------------------
    % ~   (ii) - (a) 
    % ~       [MLQB - Page 157 - Q 1]
    \item The $8^{th}$ term of the A.P. 12, 8, 4, 0, \dots is:

        \begin{multicols}{2}
        \begin{enumerate}[label=(\alph*)]
            \item -16 
            \item -12 
            \item -20 
            \item -4 
        \end{enumerate}
        \end{multicols}

    % ------- Question 1.iii ----------------------------------
    % ~   (iii) - (b)
    % ~       [MLQB - Page 140 - Q 11]
    \item The simplified form of the following is:
        \[
            \begin{bmatrix*} 
                \cos 45 \degree & \sin 30 \degree \\ 
                \sqrt{2} \cos 0 \degree & \sin 0 \degree 
            \end{bmatrix*}
            \begin{bmatrix*} 
                \sin 45 \degree & \cos 90 \degree \\ 
                \sin 90 \degree & \cot 45 \degree 
            \end{bmatrix*}
        \]

        \begin{multicols}{2}
        \begin{enumerate}[label=(\alph*)]
            \item $\begin{bmatrix*} 1 & 0 \\ 0 & 1 \end{bmatrix*}$ 
            \item $\begin{bmatrix*} 1 & \frac12 \\ 1 & 0 \end{bmatrix*}$ 
            \item $\begin{bmatrix*} 1 & 1 \\ 1 & 0 \end{bmatrix*}$ 
            \item $\begin{bmatrix*} 0 & 0 \\ 0 & 0 \end{bmatrix*}$ 
        \end{enumerate}
        \end{multicols}

    % ------- Question 1.iv ----------------------------------
    % ~   (iv) - (b)
    % ~       [MLQB - Page 130 - Q 19]
    \item Using remainder theorem, the factors of the polynomial
        $2x^3 + 3x^2 - 9x - 10$ are:

        \begin{multicols}{2}
        \begin{enumerate}[label=(\alph*)]
            \item $(x-2),(x-3),(2x+5)$ 
            \item $(x-2),(x+1),(2x+5)$ 
            \item $(x-2),(x-1),(2x-9)$ 
            \item $(x-2),(x+4),(2x-9)$ 
        \end{enumerate}
        \end{multicols}

    % ------- Question 1.v ----------------------------------
    % ~   (v) - (d)
    % ~       [MLQB - Page 45 - Q 24]
    \item Mr. Singh opened a R.D. account for 2 years and deposited
        \rupee~2,500 per month. At the time of maturity, he got 
        \rupee~67,500. The total interest earned by him during this
        period is:

        \begin{multicols}{2}
        \begin{enumerate}[label=(\alph*)]
            \item \rupee~8,500
            \item \rupee~8,000
            \item \rupee~7,000
            \item \rupee~7,500
        \end{enumerate}
        \end{multicols}

    % ------- Question 1.vi ----------------------------------
    % ~   (vi) - (d)
    % ~       [MLQB - Page 77 - Q 19]
    \item If $\dfrac{6}{x} - \dfrac{2}{x-1} = \dfrac{1}{x-2}$, then the 
        value(s) of $x$ is/are:

        \begin{multicols}{2}
        \begin{enumerate}[label=(\alph*)]
            \item $\dfrac{1}{3}$, $\dfrac{4}{3}$ 
            \item $2$, $\dfrac{1}{3}$ 
            \item $1$, $2$
            \item $3$, $\dfrac{4}{3}$ 
        \end{enumerate}
        \end{multicols}

    % ------- Question 1.vii ----------------------------------
    % ~   (vii) - (c)
    % ~       [MLQB - Page 108 - Q 26]
    \item If $\dfrac{\sqrt{5x} + \sqrt{2x-6}}{\sqrt{5x} - \sqrt{2x-6}} = 4$, then the value of $x$ is: 

        \begin{multicols}{2}
        \begin{enumerate}[label=(\alph*)]
            \item 20 
            \item 10 
            \item 30 
            \item 40 
        \end{enumerate}
        \end{multicols}

\end{enumerate}

% ======================================================================
% ~  Question 2
% ~  ----------
\newpage
\par
\noindent
\textbf{Question 2}\\
\begin{enumerate}[label=(\roman*)]

    % ~   (2.i) - 0.89 or -7.89
    % ~       [MLQB - Page 86 - Q 60]
    \item Solve $x^2 + 7x = 7$ and give your answer correct to two 
        decimal places. \hfill [4]

    % ~   (2.ii) - 90 or 60 degrees 
    % ~       [MLQB - Page 359 - Q 37.i]
    \item Solve the following equation: \hfill [4]
        \[
            \frac{\cos \theta}{1 - \sin \theta} + \frac{\cos \theta}{1 + \sin \theta} = 4
        \]

    % ~   (2.iii) - a:b = 3:2 
    % ~       [MLQB - Page 115 - Q 58]
    \item Find the value of $a:b$, given that: \hfill [5]
        \[
            \frac{a^3 + 3ab^2}{b^3 + 3a^2b} = \frac{63}{62}
        \]

\end{enumerate}
