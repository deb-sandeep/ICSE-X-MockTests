% ~ ........................................
% ~ Section B
% ~ ======================================== 
\newpage
\begin{center}
   \large
   \textbf{SECTION B (30 marks)}\\
   \vspace{5mm}
   \normalsize
   \textit{(Attempt \textbf{any three} questions from this \textbf{Section})}
\end{center}
\par

% ======================================================================
% ~  Question 4
% ~  ----------
\noindent
\textbf{Question 4}
\begin{enumerate}[label=(\roman*)]

    % ~   (4.i) - {x: 7/5 < x < 13/5, x in R}
    % ~       MLQB - 71/49.v
    \item Find the solution set of the following inequalities and draw the 
        graph of their solution sets: \hfill[3]
        \[
            \frac{3}{\abs{x-2}} > 5, x \in \mathbb{R}
        \]

    % ~   (4.ii) - 12, 14
    % ~       MLQB - 84/52
    \item Find two consecutive positive even integers whose squares have 
        the sum 340. \hfill [3]

    % ~   (4.iii) - 320, 160, 240
    % ~       MLQB - 111/42
    \item Divide \rupee~720 between Sunil, Sohil and Akhil, so that 
        Sunil gets $\dfrac45$ of Sohil's and Akhil's share together 
        and Sohil gets $\dfrac23$ of Akhil's share. \hfill [4]

\end{enumerate}

% ======================================================================
% ~  Question 5
% ~  ----------
\noindent
\textbf{Question 5}
\begin{enumerate}[label=(\roman*)]

    % ~   (5.i) - x= -2b/3a, -2b/a
    % ~       MLQB - 86/63.i
    \item Solve the quadratic equation: \hfill [3]
        \[
            3a^2x^2 + 8abx + 4b^2 = 0
        \]

    % ~   (5.ii) - Proof
    % ~       MLQB - 364/59.i
    \item Prove the identity: \hfill [3]
        \[
            \frac{1 - \cos \theta}{1 + \cos \theta} = 
            (\cot \theta - \mathrm{cosec} \ \theta)^2
        \]

    % ~   (5.iii) - a_n = log( a x b^{n-1} )
    % ~       MLQB - 163/41
    \item Determine the $n^{th}$ term of the sequence \hfill [4]
        \[
            \log(a),\ \log(ab),\ \log(ab^2),\ \log(ab^3),\ \dots
        \]

\end{enumerate}

% ======================================================================
% ~  Question 6
% ~  ----------
\newpage
\noindent
\textbf{Question 6}
\begin{enumerate}[label=(\roman*)]

    % ~   (6.i) - 40(\sqrt{3} + 1)
    % ~       MLQB - 176/5
    \item Find the sum of the first 8 terms of the following series: \hfill [3]
        \[
            1,\ \sqrt3,\ 3,\ \dots
        \]

        % ~   (6.ii) - {-3, -2, -1, 0, 1}
    % ~       MLQB - 68/iv
    \item Solve the following inequalities: \hfill [3]
        \[
            2x-5 \leq 5x+4 < 11, \ x \in \mathbb{I}
        \]

    % ~   (6.iii) - Perimeter 32 units
    % ~       MLQB - 213/15
    \item Using graph paper, plot the points $A(6,4)$ and $B(0,4)$. 
        Reflect $A$ and $B$ in the origin to get images $A'$ and $B'$. 
        Find the perimeter of figure $ABA'B'$. \hfill [4]

\end{enumerate}

% ======================================================================
% ~  Question 7
% ~  ----------
\noindent
\textbf{Question 7}
\begin{enumerate}[label=(\roman*)]

    % ~   (7.i) - b
    % ~       MLQB - 46/27
    \item Choose the right option: \hfill [4]

        \textbf{Assertion:} Mrs. Mehta has a cululative time deposit account 
        in a bank. She deposits \rupee~600 per month for 6 years and received
        \rupee~53,712 at the end of maturity period. Then the rate of interest 
        is 8\% per annum.

        \textbf{Reason:} The maturity value of a R.D. account includes the amount 
        deposited by the accoutn holder together with teh interest compounded
        quarterly at a fixed rate.

        \begin{enumerate}[label={\alph*.}]
            \setlength\itemsep{0em}
            \item Both assertion and reason are correct and reason is the
                correct explanation of assertion.
            \item Both assertion and reason are correct but reason is not
                the correct explanation of assertion.
            \item Assertion is correct but reason is not correct.
            \item Assertion is incorrect but reason is correct.
        \end{enumerate}

    % ~   (7.ii) - i) Rs. 30,000 ii) Rs. 29,160 iii) Rs. 160
    % ~       MLQB - 35/32
    \item A wholesaler buys a TV from the manufacturer for \rupee~25,000. 
        He marks the price of the TV 20\% above his cost price and sells it to 
        a retailer at a 10\% discount on the marked price. If the rate of 
        GST is 8\%, find: \hfill [6]
        \begin{enumerate}
            \setlength\itemsep{0em}
            \item Marked price
            \item Retailer's cost price inclusive of tax
            \item GST paid by the wholesaler
        \end{enumerate}

\end{enumerate}


