% ~ Section B
\newpage
\begin{center}
   \large
   \textbf{SECTION B (20 marks)}\\
   \vspace{5mm}
   \normalsize
   \textit{(Attempt \textbf{any two} questions from this \textbf{Section})}
\end{center}
\par

% ======================================================================
% ~  Question 3
\noindent
\textbf{Question 3}
\begin{enumerate}[label=(\roman*)]

    % ~   (3.i) - -2 <= x < 2.3 
    % ~       [MLQB - Page 72 - Q 50]
    \item Solve the following inequalities: \hfill [3]
        \[
            -\frac{x}{3} - 4 \leq \frac{x}{2} - \frac{7}{3} < -\frac{7}{6}, x \in \mathbb{R}
        \]
        Represent the solution set on a number line.

    % ~   (3.ii) - -2 
    % ~       [MLQB - Page 89 - Q 71]
    \item Solve for $x$: \hfill [3]
        \[
            9^{x+2} - 6.3^{x+1} + 1 = 0 
        \]

    % ~   (3.iii) - x:y = 2:3 
    % ~       [MLQB - Page 109 - Q 32]
    \item Given: \hfill [4]
        \[
            \frac{x^3 + 12x}{6x^2 + 8} = \frac{y^3 + 27y}{9y^2 + 27}
        \]
        Using componendo and dividendo find $x:y$.

\end{enumerate}

% ======================================================================
% ~  Question 4
\noindent
\textbf{Question 4}
\begin{enumerate}[label=(\roman*)]

    % ~   (4.i) - 3 years 
    % ~       [MLQB - Page 43 - Q 17]
    \item The matured value of a RD account is \rupee~16,176. 
        If the monthly installment is \rupee~400 and the rate 
        of interest is 8\% p.a., what is the time period of
        this R.D. account? \hfill [3]

    % ~   (4.ii) - Proof 
    % ~       [MLQB - Page 369 - Q 77]
    \item If $\cos \theta + \sin \theta = \sqrt{2} \cos \theta$, show that: \hfill [3]
        \[
            \cos \theta - \sin \theta = \sqrt{2} \sin \theta
        \]

    % ~   (4.iii) - {} 
    % ~       [Balal paper - Q 35]
    \item Solve and graph the solution on a number line: \hfill [4]
        \[
            \frac{3x}{5} - \frac{2x-1}{3} > 1, x \in \mathbb{W}
        \]

\end{enumerate}

% ======================================================================
% ~  Question 5
\newpage
\noindent
\textbf{Question 5}
\begin{enumerate}[label=(\roman*)]

    % ~   (5.i) - {x: -5 < x <= 2, x in R} 
    % ~       [MLQB - Page 69 - Q 42]
    \item Solve the following ineqation and represent the solution 
        set on the number line: \hfill [4]
        \[
            \frac{3x}{5} + 2 < x + 4 \leq \frac{x}{2} + 5, x \in \mathbb{R}
        \]

    % ~   (5.ii) - 22.214
    % ~       [MLQB - Page 403 - Q 57]
    \item Using the step deviation method, find the arithmetic mean of the 
        distribution: \hfill [6]
        \begin{table}[h]
        \centering
        \renewcommand{\arraystretch}{1.3}
        \begin{tabularx}{0.9\textwidth}{| p {2.75 cm} | X | X | X | X | X | X | X | X | X | X | }
            \hline
            Variate (x) & 5 & 10 & 15 & 20 & 25 & 30 & 35 & 40 & 45 & 50 \\
            \hline
            Frequenty (f) & 20 & 43 & 75 & 67 & 72 & 45 & 39 & 9 & 8 & 6 \\
            \hline
        \end{tabularx}
        \end{table}
        

\end{enumerate}


