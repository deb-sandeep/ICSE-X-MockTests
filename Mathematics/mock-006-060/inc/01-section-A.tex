% ~ Section A
\begin{center}
   \large
   \textbf{SECTION A (20 marks)}\\
   \vspace{5mm}
   \normalsize
   \textit{(Attempt \textbf{all} questions from this \textbf{Section})}
\end{center}
\par
\noindent
% ~  Question 1
\textbf{Question 1}\\
Choose the correct answers to the questions from the given options. \hfill [7]\\
(Do not copy the questions, write the correct answers only.)
\par
\vspace{2mm}
\begin{enumerate}[label=(\roman*)]

    % ------- Question 1.i ----------------------------------
    % ~   (i) - (c) 5:3
    % ~       [MLQB - Page 104 - Q 9]
    \item If $(2x^2 - 5y^2) : xy = 1:3$, then $x:y$ is: 

        \begin{multicols}{2}
        \begin{enumerate}[label=(\alph*)]
            \item 3:2 
            \item -5:3
            \item 5:3
            \item -3:2 
        \end{enumerate}
        \end{multicols}

    % ------- Question 1.ii ----------------------------------
    % ~   (ii) - (b) 
    % ~       [MLQB - Page 142 - Q 22]
    \newpage
    \item If $A = \begin{bmatrix*} 7 & 3 \\ 5 & 2 \end{bmatrix*}$ and 
        $B = \begin{bmatrix*} 2 & 5 \\ 4 & 5 \end{bmatrix*}$, then the 
        matrix $C$ such that $2A + 3C = 8B$, is:

        \begin{multicols}{2}
        \begin{enumerate}[label=(\alph*)]
            \item $\begin{bmatrix*} 2 & 24 \\ 22 & 36 \end{bmatrix*}$
            \item $\begin{bmatrix*} \frac{2}{3} & \frac{34}{3} \\ \frac{22}{3} & \frac{36}{3} \end{bmatrix*}$
            \item $\begin{bmatrix*} \frac{2}{3} & \frac{22}{3} \\ \frac{36}{3} & \frac{34}{3} \end{bmatrix*}$
            \item $\begin{bmatrix*} 1 & 17 \\ 11 & 18 \end{bmatrix*}$
        \end{enumerate}
        \end{multicols}

    % ------- Question 1.iii ----------------------------------
    % ~   (iii) - (d) (x+1),(x+2),(x-2)
    % ~       [MLQB - Page 129 - Q 17]
    \item Using the remainder theorem, the factors of the polynomial
        $x^3 + x^2 - 4x - 4$ are:

        \begin{multicols}{2}
        \begin{enumerate}[label=(\alph*)]
            \item $(x+1),(x-2),(x-2)$ 
            \item $(x-1),(x+1),(x+2)$ 
            \item $(x+1),(x+1),(x-2)$ 
            \item $(x+1),(x+2),(x-2)$ 
        \end{enumerate}
        \end{multicols}

    % ------- Question 1.iv ----------------------------------
    % ~   (iv) - (b)
    % ~       [MLQB - Page 63 - Q 16]
    \item The set of values of $x$, satisfying both $7x + 3 \geq 3x - 5$ and 
        $\dfrac{x}{4} - 5 \leq \dfrac{5}{4} - x, x \in \mathbb{N}$ is:

        \begin{multicols}{2}
        \begin{enumerate}[label=(\alph*)]
            \item \{ -2, -1, 0, 1, 2, 3, 4, 5 \} 
            \item \{ 1, 2, 3, 4, 5 \} 
            \item \{ 0, 1, 2, 3, 4, 5 \} 
            \item None of the above
        \end{enumerate}
        \end{multicols}

    % ------- Question 1.v ----------------------------------
    % ~   (v) - (c)
    % ~       [MLQB - Page 77 - Q 16]
    \item If $(x+1)(2x+8) = (x+7)(x+3)$, the using factorisation method, the value 
        of $x$ are:

        \begin{multicols}{2}
        \begin{enumerate}[label=(\alph*)]
            \item $\sqrt{12}$, $\sqrt{13}$ 
            \item $-\sqrt{13}$, $-\sqrt{13}$ 
            \item $\pm \sqrt{13}$
            \item $\sqrt{13}$, $\sqrt{13}$ 
        \end{enumerate}
        \end{multicols}

    % ------- Question 1.vi ----------------------------------
    % ~   (vi) - (d)
    % ~       [MLQB - Page 354 - Q 9]
    \item Evaluate: $\dfrac{\sec A}{\sec A - 1} + \dfrac{\sec A}{\sec A + 1}$ 

        \begin{multicols}{2}
        \begin{enumerate}[label=(\alph*)]
            \item $(1 + \mathrm{cosec} \ A )$ 
            \item $(2 + \sec A)$ 
            \item $3 \sec A$ 
            \item $2 \ \mathrm{cosec}^2 A$ 
        \end{enumerate}
        \end{multicols}

    % ------- Question 1.vii ----------------------------------
    % ~   (vii) - (b)
    % ~       [MLQB - Page 389 - Q 2]
    \newpage
    \item Following is the distribution of monthly wages of 200 employees in a factory: 

        \begin{table}[h]
        \centering
        \renewcommand{\arraystretch}{1.3}
        \begin{tabularx}{0.5\textwidth}{| p {3 cm} | X | }
            \hline
            \rowcolor{lightgray!30} Wages (\rupee) & No. of workers \\
            \hline
            80 – 100  & 20 \\
            \hline
            100 – 120 & 30 \\
            \hline
            120 – 140 & 20 \\
            \hline
            140 – 160 & 40 \\
            \hline
            160 – 180 & 90 \\
            \hline
        \end{tabularx}
        \end{table}

        Calculate the average income of the employees:

        \begin{multicols}{2}
        \begin{enumerate}[label=(\alph*)]
            \item 150 
            \item 145
            \item 140
            \item 135
        \end{enumerate}
        \end{multicols}

\end{enumerate}

% ======================================================================
% ~  Question 2
\par
\noindent
\textbf{Question 2}\\
\begin{enumerate}[label=(\roman*)]

    % ~   (2.i) - Proof
    % ~       [MLQB - Page 365 - Q 61.ii]
    \item Prove that \hfill [4]
        \[
            \frac{\sin \theta - 2 \sin^3 \theta}{2 \cos^3 \theta - \cos \theta} = \tan \theta
        \]

    % ~   (2.ii) - p=2, q=2 
    % ~       [MLQB - Page 145 - Q 38]
    \item Given \hfill [4]
        \[
            A = \begin{bmatrix*}  2 & -6 \\ 2 & 0 \end{bmatrix*},
            B = \begin{bmatrix*} -3 &  2 \\ 4 & 0 \end{bmatrix*},
            C = \begin{bmatrix*}  4 &  0 \\ 0 & 2 \end{bmatrix*}
        \]
        Determine the matrix $X$, such that
        \[
            A + 2X = 2B + C
        \]

    % ~   (2.iii) - (x-1)(x+2)(2x+1)
    % ~       [MLQB - Page 135 - Q 48]
    \item The expression $2x^3 + ax^2 + bx - 2$ leaves the remainder
        7 and 0 when divided by $(2x-3)$ and $(x+2)$ respectively. 
        
        Calculate the value of $a$, and $b$.

        With these values of $a$ and $b$, factorise the expression 
        completely.\hfill [5]

\end{enumerate}
