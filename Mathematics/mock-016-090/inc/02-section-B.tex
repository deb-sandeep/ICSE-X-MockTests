% ~ ........................................
% ~ Section B
% ~ ======================================== 
\par\noindent\rule{\textwidth}{0.4pt}
\begin{center}
   \large
   \textbf{SECTION B (30 marks)}\\
   \vspace{5mm}
   \normalsize
   \textit{(Attempt \textbf{any three} questions from this \textbf{Section})}
\end{center}
\par

% ======================================================================
% ~  Question 3
% ~  ----------
\noindent
\textbf{Question 3}
\begin{enumerate}[label=(\roman*)]

    % ~   (3.i) - 0 or 60 degree 
    % ~       [MLQB - Page 359 - Q 37.ii]
    \item Solve for $\theta$ \hfill [3]
        \[
            \frac{\cos^2 \theta - 3 \cos \theta + 2}{\sin^2 \theta} = 1
        \]

    % ~   (3.ii) - r=1/3
    % ~       [EC - Page 146 - Q 23]
    \item If $x$, $2y$, $3z$ are in A.P., where the distinct numbers 
        $x$, $y$, $z$ are in G.P., then what is the common ratio of the G.P.?
        \hfill [3]

    % ~   (3.iii) - 9735 m^2
    % ~       [EC - Page 309 - Q 36]
    \item A circus tent is cylindrical to a height of 3 m and conical 
        above it. If its base radius is 52.5 m and the slant height of 
        the conical portion is 53 m, find the area of canvas required 
        to make the tent. \hfill [4]

\end{enumerate}

% ======================================================================
% ~  Question 4
% ~  ----------
\newpage
\noindent
\textbf{Question 4}
\begin{enumerate}[label=(\roman*)]

    % ~   (4.i) - x = 1,2 
    % ~       [MLQB - Page 79 - Q 24]
    \item Find the roots of the following equation. \hfill [3]
        \[
            \frac{1}{x+4} - \frac{1}{x-7} = \frac{11}{30},\ x \ne -4, 7
        \]

    % ~   (4.ii) - x = 25
    % ~       [MLQB - Page 110 - Q 34]
    \item If $(x-9) : (3x+6)$ is the duplicate ratio of $4:9$, 
        find the value of $x$. \hfill [3]

    % ~   (4.iii) - 8
    % ~       [EC - Page 119 - Q 2]
    \item If $\begin{bmatrix*} (2p+q) & (p-2q) \\ (5r-s) & (4r+3s) \end{bmatrix*} = 
        \begin{bmatrix*} 4 & -3 \\ 11 & 24 \end{bmatrix*}$, then the 
        value of $p + q - r + 2s$ is? \hfill [4]
\end{enumerate}
\vspace{10pt}

% ======================================================================
% ~  Question 5
% ~  ----------
\noindent
\textbf{Question 5}
\begin{enumerate}[label=(\roman*)]

    % ~   (4.i) - 32nd term
    % ~       [MLQB - Page 164 - Q 45]
    \item Determine which term of the A.P. 121, 117, 113, \dots is 
        its first negative term? \hfill [3]

    % ~   (4.ii) - k = -2
    % ~       [EC - Page 112 - Q 31]
    \item Using the remainder theorem, find the remainders obtained when 
        $x^3 + (kx+8)x + k$ is divided by $(x+1)$ and $(x-2)$. Hence, 
        find $k$ if the sum of the two remainders is 1. \hfill [3]

    % ~   (4.iii) - Roots are sqrt{6} and -sqrt{6}/3 
    % ~       [EC - Page 63 - Q 7.b]
    \item Find the roots of the following equation by using 
        the quadratic formula. \hfill [4]
        \[
            \sqrt{3}x^2 - 2\sqrt{2}x - 2\sqrt{3} = 0
        \]

\end{enumerate}

% ======================================================================
% ~  Question 6
% ~  ----------
\noindent
\textbf{Question 6}
\begin{enumerate}[label=(\roman*)]

    % ~   (5.i) - 10% p.a.
    % ~       [EC - Page 33 - Q 35]
    \item Harshit has a cumulative bank account and deposits \rupee~600 
        per month for a period of 4 years. If he gets \rupee~5880 as 
        interest at the time of maturity, find the rate of interest 
        per annum. \hfill [4]

    % ~   (5.ii) - Proof
    % ~       [EC - Page 101 - Q 47]
    \item If $x = \dfrac{\sqrt{2a+1} + \sqrt{2a-1}}{\sqrt{2a+1} - \sqrt{2a-1}}$, 
        then prove that $x^2 - 4ax + 1 = 0$. \hfill [6]

\end{enumerate}


