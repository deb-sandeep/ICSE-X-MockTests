% ~ ........................................
% ~ Section A
% ~ ======================================== 
\begin{center}
   \large
   \textbf{SECTION A (20 marks)}\\
   \vspace{5mm}
   \normalsize
   \textit{(Attempt \textbf{all} questions from this \textbf{Section})}
\end{center}
\par
\noindent
% ~  Question 1
% ~  ----------
\textbf{Question 1}\\
Choose the correct answers to the questions from the given options. \hfill [7]\\
(Do not copy the questions, write the correct answers only.)
\par
\vspace{2mm}
\begin{enumerate}[label=(\roman*)]

    % ------- Question 1.i ----------------------------------
    % ~   (i) - (d) 
    % ~       [EC - Page 27 - Q 11]
    \item Mr. Sharma deposited \rupee~500 every month in a cumulative deposit
        account for 2 years. If the bank pays interest at the rate of 7\% per
        annum, then the amount he gets on maturity is:

        \begin{multicols}{2}
        \begin{enumerate}[label=(\alph*)]
            \item \rupee~8,750 
            \item \rupee~6,875
            \item \rupee~10,875 
            \item \rupee~12,875 
        \end{enumerate}
        \end{multicols}

    % ------- Question 1.ii ----------------------------------
    % ~   (ii) - (b) 
    % ~       [EC - Page 94 - Q 202]
    \newpage
    \item If $\dfrac{\sqrt{3x+4} + \sqrt{3x-5}}{\sqrt{3x+4} - \sqrt{3x-5}} = 9$, 
        then, the value of x is:

        \begin{multicols}{2}
        \begin{enumerate}[label=(\alph*)]
            \item 2
            \item 7
            \item 4
            \item 9
        \end{enumerate}
        \end{multicols}

    % ------- Question 1.iii ----------------------------------
    % ~   (iii) - (d)
    % ~       [EC - Page 106 - Q 5]
    \item What is the remainder when $2x^2 + 8x -5$ is divided by $(x-2)$?

        \begin{multicols}{2}
        \begin{enumerate}[label=(\alph*)]
            \item 17 
            \item 16 
            \item 18 
            \item 19 
        \end{enumerate}
        \end{multicols}

    % ------- Question 1.iv ----------------------------------
    % ~   (iv) - (d)
    % ~       [MLQB - Page 140 - Q 10]
    \item If $A = \begin{bmatrix*} -1 & 1 \\ a & b \end{bmatrix*}$ and $A^2 = I$,
        then the values of $a$ and $b$ are:

        \begin{multicols}{2}
        \begin{enumerate}[label=(\alph*)]
            \item $a=b=1$ 
            \item $a=1, b=0$ 
            \item $a=b=0$ 
            \item $a=0, b=1$ 
        \end{enumerate}
        \end{multicols}

    % ------- Question 1.v ----------------------------------
    % ~   (v) - (c)
    % ~       [EC - Page 302 - Q 6]
    \item A cylinder and a cone are of the same base and of same height. The 
        ratio of their volumes is:

        \begin{multicols}{2}
        \begin{enumerate}[label=(\alph*)]
            \item $1:2$
            \item $2:1$
            \item $3:1$
            \item $3:2$
        \end{enumerate}
        \end{multicols}

    % ------- Question 1.vi ----------------------------------
    % ~   (vi) - (a)
    % ~       [EC - Page 325 - Q 4]
    \item $\sqrt{\dfrac{1 + \cos A}{1 - \cos A}} = $

        \begin{multicols}{2}
        \begin{enumerate}[label=(\alph*)]
            \item $\mathrm{cosec} \ A + \cot A$ 
            \item $\mathrm{cosec} \ A - \cot A$ 
            \item $\mathrm{cosec} \ A \cot A$ 
            \item None of these
        \end{enumerate}
        \end{multicols}

    % ------- Question 1.vii ----------------------------------
    % ~   (vii) - (d)
    % ~       [MLQB - Page 322 - Q 5]
    \item In a cylinder, if the radius is one-fourth and height is halved,
        then the volume will be:

        \begin{multicols}{2}
        \begin{enumerate}[label=(\alph*)]
            \item doubled 
            \item $\dfrac18$ times
            \item $\dfrac{1}{16}$ times
            \item $\dfrac{1}{32}$ times
        \end{enumerate}
        \end{multicols}

\end{enumerate}

% ======================================================================
% ~  Question 2
% ~  ----------
\newpage
\par
\noindent
\textbf{Question 2}
\begin{enumerate}[label=(\roman*)]

    % ~   (2.i) - k = 8
    % ~       [MLQB - Page 133 - Q 37]
    \item When $x^3 + 3x^2 - kx + 4$ is divided by $(x-2)$, 
        the remainder is $k$. Find the value of $k$. \hfill [4]

    % ~   (2.ii) - Proof 
    % ~       [EC - Page 332 - Q 29]
    \item Prove that: \hfill [4]
        \[
            \frac{\sin \theta - 2 \sin^3 \theta}
                 {2 \cos^3 \theta - \cos \theta} 
            \times \cot \theta = 1
        \]

    % ~   (2.iii) - r2 = 3, surface area = 595.2 cm^2 
    % ~       [MLQB - Page 332 - Q 58]
    \item A solid metallic sphere of radius 6 cm is melted 
        and made into a solid cylinder of height 32 cm. Find 
        the: \hfill [5]
        \begin{enumerate}
            \item Radius of the cylinder
            \item Curved surface area of the cylinder
        \end{enumerate}

\end{enumerate}

