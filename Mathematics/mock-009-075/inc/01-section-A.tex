% ~ ........................................
% ~ Section A
% ~ ======================================== 
\begin{center}
   \large
   \textbf{SECTION A (20 marks)}\\
   \vspace{5mm}
   \normalsize
   \textit{(Attempt \textbf{all} questions from this \textbf{Section})}
\end{center}
\par
\noindent
% ~  Question 1
% ~  ----------
\textbf{Question 1}\\
Choose the correct answers to the questions from the given options. \hfill [7]\\
(Do not copy the questions, write the correct answers only.)
\par
\vspace{2mm}
\begin{enumerate}[label=(\roman*)]

    % ------- Question 1.i ----------------------------------
    % ~   (i) - (a) 
    % ~       [MLQB - Page 41 - Q 4]
    \item Mukesh deposited \rupee~150 per month in a recurring deposit
        account for 2 years. Find the amount payable to him on maturity if the 
        rate of interest is 8\% per annum.

        \begin{multicols}{2}
        \begin{enumerate}[label=(\alph*)]
            \item \rupee~3900 
            \item \rupee~4200
            \item \rupee~4500
            \item \rupee~5000
        \end{enumerate}
        \end{multicols}

    % ------- Question 1.ii ----------------------------------
    % ~   (ii) - (b) 
    % ~       [MLQB - Page 65 - Q 24]
    \newpage
    \item To solve the linear inequation $5x + 7 < 27, x \in \mathbb{I}$, we add
        -7 to both sides. With this opeartion, the sign of inequality.

        \begin{multicols}{2}
        \begin{enumerate}[label=(\alph*)]
            \item Reverses 
            \item Remains same
            \item Data insufficient
            \item None of the above
        \end{enumerate}
        \end{multicols}

    % ------- Question 1.iii ----------------------------------
    % ~   (iii) - (b)
    % ~       [MLQB - Page 77 - Q 17]
    \item The two natural numbers which differ by 3 and whose squares
        have the sum 117, are:

        \begin{multicols}{2}
        \begin{enumerate}[label=(\alph*)]
            \item 4,7 
            \item -6, 9
            \item 8, 11
            \item 5, 8
        \end{enumerate}
        \end{multicols}

    % ------- Question 1.iv ----------------------------------
    % ~   (iv) - (c)
    % ~       [MLQB - Page 141 - Q 16]
    \item If $A = \begin{bmatrix*} 1 & 1 \\ 8 & 3 \end{bmatrix*}$, then 
        $A^2 - 4A =$

        \begin{multicols}{2}
        \begin{enumerate}[label=(\alph*)]
            \item $\begin{bmatrix*} -3 & -3 \\ 60 &  5 \end{bmatrix*}$ 
            \item $\begin{bmatrix*} -3 & -3 \\  8 &  3 \end{bmatrix*}$ 
            \item $\begin{bmatrix*}  5 &  0 \\  0 &  5 \end{bmatrix*}$ 
            \item $\begin{bmatrix*}  8 &  3 \\ 24 & 14 \end{bmatrix*}$ 
        \end{enumerate}
        \end{multicols}

    % ------- Question 1.v ----------------------------------
    % ~   (v) - (b)
    % ~       [MLQB - Page 356 - Q 24]
    \item If $\sin \theta = \dfrac{7}{15}$, find the value of $(1 + \tan^2 \theta)$. 

        \begin{multicols}{2}
        \begin{enumerate}[label=(\alph*)]
            \item $\dfrac{24}{25}$ 
            \item $\dfrac{625}{576}$ 
            \item $\dfrac{49}{625}$ 
            \item None of these 
        \end{enumerate}
        \end{multicols}

    % ------- Question 1.vi ----------------------------------
    % ~   (vi) - (d)
    % ~       [MLQB - Page 128 - Q 10]
    \item Given that $(x+2)$ and $(x+4)$ are the factors of 
        $3x^3 + ax^2 - 6x -b$. The values of $a$ and $b$ respectively
        are:

        \begin{multicols}{2}
        \begin{enumerate}[label=(\alph*)]
            \item 4,2
            \item 2,4
            \item 40, 13
            \item 13, 40
        \end{enumerate}
        \end{multicols}

    % ------- Question 1.vii ----------------------------------
    % ~   (vii) - (d)
    % ~       [MLQB - Page 106 - Q 20]
    \item If $\dfrac{x^2+2x}{2x+4} = \dfrac{y^2+3y}{3y+9}$, then the 
        value of $2x:3y$ is: 

        \begin{multicols}{2}
        \begin{enumerate}[label=(\alph*)]
            \item $16:27$ 
            \item $1:1$ 
            \item $2:3$ 
            \item $4:9$ 
        \end{enumerate}
        \end{multicols}

\end{enumerate}

% ======================================================================
% ~  Question 2
% ~  ----------
\newpage
\par
\noindent
\textbf{Question 2}\\
\begin{enumerate}[label=(\roman*)]

    % ~   (2.i)  0 or 30 degrees-
    % ~       [MLQB - Page 359 - Q 35]
    \item Solve: $2 \cos^2 \theta + \sin \theta - 2 = 0$ \hfill [4]

    % ~   (2.ii) - 129.9 m
    % ~       [MLQB - Page 359 - Q 34]
    \item The string of a kite is 150 m long and it makes an angle 
        of 60\degree \ with the horizontal. Determine the height 
        of the kite from the ground. \hfill [4]

    % ~   (2.iii) - Mean wage = 1512 
    % ~       [MLQB - Page 402 - Q 56]
    \item Find the mean wage of a worker from the following data: \hfill [5]
        \begin{table}[h]
        \centering
        \renewcommand{\arraystretch}{1.3}
        \begin{tabularx}{0.5\textwidth}{| p {3 cm} | X | }
            \hline
            \rowcolor{lightgray!30} Wages (in \rupee) & Number of workers \\
            \hline
             1400  & 15  \\
            \hline
             1450  & 20 \\
            \hline
             1500  & 18 \\
            \hline
             1550  & 27 \\
            \hline
             1600  & 15 \\
            \hline
             1650  & 3 \\
            \hline
             1700  & 2 \\
            \hline
        \end{tabularx}
        \end{table}

\end{enumerate}
