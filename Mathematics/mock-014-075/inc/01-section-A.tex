% ~ ........................................
% ~ Section A
% ~ ======================================== 
\begin{center}
   \large
   \textbf{SECTION A (20 marks)}\\
   \vspace{5mm}
   \normalsize
   \textit{(Attempt \textbf{all} questions from this \textbf{Section})}
\end{center}
\par
\noindent
% ~  Question 1
% ~  ----------
\textbf{Question 1}\\
Choose the correct answers to the questions from the given options. \hfill [7]\\
(Do not copy the questions, write the correct answers only.)
\par
\vspace{2mm}
\begin{enumerate}[label=(\roman*)]

    % ------- Question 1.i ----------------------------------
    % ~   (i) - (a) 
    % ~       [MLQB - Page 355 - Q 16]
    \item If $\sin \theta + \sin^2 \theta = 1$, then $\cos^2 \theta + \cos^4 \theta$ is:

        \begin{multicols}{2}
        \begin{enumerate}[label=(\alph*)]
            \item 1 
            \item $\dfrac12$ 
            \item 2 
            \item 4 
        \end{enumerate}
        \end{multicols}

    % ------- Question 1.ii ----------------------------------
    % ~   (ii) - (b) 
    % ~       [MLQB - Page 175 - Q 2]
    \item If $k$, $2(k+1)$, $3(k+1)$ are consecutive terms of a G.P., then the 
        value of $k$ is:

        \begin{multicols}{2}
        \begin{enumerate}[label=(\alph*)]
            \item -1 
            \item -4 
            \item 1 
            \item 4 
        \end{enumerate}
        \end{multicols}

    % ------- Question 1.iii ----------------------------------
    % ~   (iii) - (a)
    % ~       [MLQB - Page 158 - Q 11]
    \item If 8 times the eighth term of an A.P. is 15 times the fifteenth term,
        then the 23\textsuperscript{rd} term of the A.P. is:

        \begin{multicols}{2}
        \begin{enumerate}[label=(\alph*)]
            \item 0 
            \item 22
            \item 23
            \item 15
        \end{enumerate}
        \end{multicols}

    % ------- Question 1.iv ----------------------------------
    % ~   (iv) - (b)
    % ~       [MLQB - Page 140 - Q 14]
    \item Find the value of $x$ and $y$ if:
        \[
            2 \begin{bmatrix*} 3 & 4 \\ 5 & x \end{bmatrix*} + 
              \begin{bmatrix*} 1 & y \\ 0 & 1 \end{bmatrix*} = 
              \begin{bmatrix*} 7 & 0 \\ 10 & 5 \end{bmatrix*}
        \]

        \begin{multicols}{2}
        \begin{enumerate}[label=(\alph*)]
            \item $x=4$, $y=-4$ 
            \item $x=2$, $y=-8$ 
            \item $x=2$, $y=-4$ 
            \item $x=4$, $y=-8$ 
        \end{enumerate}
        \end{multicols}

    % ------- Question 1.v ----------------------------------
    % ~   (v) - (c)
    % ~       [MLQB - Page 78 - Q 22]
    \item If the sum of two sides, other than hypotenuese of a 
        right-angled triangle is 17 cm and the perimeter is 
        30 cm, then the lenghts of the other two sides are:

        \begin{multicols}{2}
        \begin{enumerate}[label=(\alph*)]
            \item 7 cm, 10 cm 
            \item 4 cm, 13 cm 
            \item 5 cm, 12 cm 
            \item 6 cm, 11 cm 
        \end{enumerate}
        \end{multicols}

    % ------- Question 1.vi ----------------------------------
    % ~   (vi) - (b)
    % ~       [MLQB - Page 64 - Q 18]
    \item The solution set of $-2 + 10x \leq 13x + 10 < 24 + 10x, x \in \mathbb{Z}$ 
        is:

        \begin{enumerate}[label=(\alph*)]
            \item \{ -4, -3, -2, -1, 0, 1, 2, 3, 4, 5 \}
            \item \{ -4, -3, -2, -1, 0, 1, 2, 3, 4 \}
            \item \{ -3, -2, -1, 0, 1, 2, 3, 4 \}
            \item \{ -3, -2, -1, 0, 1, 2, 3, 4, 5 \}
        \end{enumerate}

    % ------- Question 1.vii ----------------------------------
    % ~   (vii) - (d)
    % ~       [MLQB - Page 41 - Q 5]
    \item Satyam deposited \rupee~200 per month in a recurring deposit account 
        for 18 months. If the rate of interest is 9\% per annum, then the 
        interest earned by him during this period is:

        \begin{multicols}{2}
        \begin{enumerate}[label=(\alph*)]
            \item \rupee~3,856.50 
            \item \rupee~3,343.50 
            \item \rupee~330
            \item \rupee~256.50 
        \end{enumerate}
        \end{multicols}

\end{enumerate}

% ======================================================================
% ~  Question 2
% ~  ----------
\newpage
\par
\noindent
\textbf{Question 2}\\
\begin{enumerate}[label=(\roman*)]

    % ~   (2.i) - 45 degree
    % ~       [MLQB - Page 358 - Q 32]
    \item Given that \hfill [4]
        \[
            \tan( \theta_1 + \theta_2 ) = 
            \frac{\tan \theta_1 + \tan \theta_2}{1 - \tan \theta_1 \tan \theta_2}
        \]
        Determine $(\theta_1 + \theta_2)$, when $\tan \theta_1 = \dfrac12$ and
        $\tan \theta_2 = \dfrac13$.

    % ~   (2.ii) - -13, -8 and -3 
    % ~       [MLQB - Page 174 - Q 10]
    \item The sum of 4\textsuperscript{th} and 8\textsuperscript{th} terms 
        of an A.P. is 24 and the sum of the 6\textsuperscript{th} and 
        10\textsuperscript{th} terms is 44. Find the first three terms of 
        the A.P. \hfill [4]

    % ~   (2.iii) - r = 21 cm, vol = 34650 cm3
    % ~       [MLQB - Page 328 - Q 32]
    \item The circumference of the base of a cylindrical vessel is 132 cm
        and its height is 25 cm. Find the radius and volume of the cylinder. \hfill [5]

\end{enumerate}
