% ~ Section A
\begin{center}
   \large
   \textbf{SECTION A (20 marks)}\\
   \vspace{5mm}
   \normalsize
   \textit{(Attempt \textbf{all} questions from this \textbf{Section})}
\end{center}
\par
\noindent
% ~  Question 1
\textbf{Question 1}\\
Choose the correct answers to the questions from the given options. \hfill [7]\\
(Do not copy the questions, write the correct answers only.)
\par
\vspace{2mm}
\begin{enumerate}[label=(\roman*)]

    % ------- Question 1.i ----------------------------------
    % ~   (i) - (d) Rs. 2650
    % ~       [MLQB - Page 41 - Q7]
    \item Seema deposited \rupee 100 per month for 24 months in a bank's recurring
        deposit account. If the bank pays an interest of 10\% p.a., then the amount 
        she gets on maturity is:

        \begin{multicols}{2}
        \begin{enumerate}[label=(\alph*)]
            \item \rupee 1,490 
            \item \rupee 1,940
            \item \rupee 2,065
            \item \rupee 2,650
        \end{enumerate}
        \end{multicols}

    % ------- Question 1.ii ----------------------------------
    % ~   (ii) - (c) 
    % ~       [MLQB - Page 61 - Q3]
    \newpage
    \item If $2x-3 < x+1 < 4x+7, x \in \mathbb{R}$, then the solution set of $x$ on the
       number line is: 

       \img{9cm}{A-1-ii}{Number lines for question (ii)}{}

    % ------- Question 1.iii ----------------------------------
    % ~   (iii) - (c) 2/3, -1 
    % ~       [MLQB - Page 76 - Q13]
    \item The value of $k$ for which the equation $x^2 + 4kx + (k^2-k+2) = 0$ has equal 
        roots, is:

        \begin{multicols}{2}
        \begin{enumerate}[label=(\alph*)]
            \item -1, $-\frac{2}{3}$
            \item -$\frac{2}{3}$, 1
            \item $\frac{2}{3}$, -1
            \item 1, $\frac{2}{3}$
        \end{enumerate}
        \end{multicols}

    % ------- Question 1.iv ----------------------------------
    % ~   (iv) - (a) p=r
    % ~       [MLQB - Page 127 - Q7]
    \item If both $(x-2)$ and $\left(x-\frac{1}{2}\right)$ are the factors of $px^2 + 5x + r$, then:

        \begin{multicols}{2}
        \begin{enumerate}[label=(\alph*)]
            \item $p=r$
            \item $p=2r$
            \item $2p=r$
            \item $p=r+2$
        \end{enumerate}
        \end{multicols}

    % ------- Question 1.v ----------------------------------
    % ~   (v) - (c) x=0, y=10
    % ~       [MLQB - Page 139 - Q9]
    \item If $\begin{bmatrix} 1 & 2 \\ 2 & 9 \end{bmatrix}
              \begin{bmatrix} x \\ y \end{bmatrix} = 
              \begin{bmatrix} 20 \\ 90 \end{bmatrix}$, then the values of $x$ and $y$ are:

        \begin{multicols}{2}
        \begin{enumerate}[label=(\alph*)]
            \item $x=10, y=10$
            \item $x=5, y=4$
            \item $x=0, y=10$
            \item $x=4, y=5$
        \end{enumerate}
        \end{multicols}

    % ------- Question 1.vi ----------------------------------
    % ~   (vi) - (b) 1 + sec(theta)
    % ~       [MLQB - Page 353 - Q6]
    \item Find the value of \[\frac{\sin \theta \tan \theta}{1 - \cos \theta}\]

        \begin{multicols}{2}
        \begin{enumerate}[label=(\alph*)]
            \item $1 - \sin \theta$ 
            \item $1 + \sec \theta$ 
            \item $1 + \sin \theta$ 
            \item $1 - \sec \theta$ 
        \end{enumerate}
        \end{multicols}

    % ------- Question 1.vii ----------------------------------
    % ~   (vii) - (c) 52.4 
    % ~       [MLQB - Page 390  - Q8]
    \newpage
    \item Find the average of the following distributions: 

        \begin{table}[h]
        \centering
        \renewcommand{\arraystretch}{1.3}
        \begin{tabularx}{0.5\textwidth}{| p {2.75 cm} | X | X | X | X | X | }
            \hline
            Variate (x) & 10 & 20 & 50 & 70 & 90\\
            \hline
            Frequenty (f) & 8 & 10 & 10 & 12 & 10\\
            \hline
        \end{tabularx}
        \end{table}

        \begin{multicols}{2}
        \begin{enumerate}[label=(\alph*)]
            \item 47.5
            \item 51.7
            \item 52.4
            \item 52.8
        \end{enumerate}
        \end{multicols}

\end{enumerate}

% ======================================================================
% ~  Question 2
\par
\noindent
\textbf{Question 2}\\
\begin{enumerate}[label=(\roman*)]

    % ~   (2.i) - 36 months
    % ~       [MLQB - Page 47 - Q33]
    \item Priyanka has a recurring deposit account of \rupee 1000 per month
        at 10\% per annum. If she gets \rupee 5550 as interest at the time 
        of maturity, find the total time for which the account was held. \hfill [4]

    % ~   (2.ii) - Proof
    % ~       [MLQB - Page 364 - Q59(ii)]
    \item Prove the following identities: 
        $ \dfrac{1 - \tan^2 \theta}{\cot^2 \theta - 1} = \tan^2 \theta $ \hfill [4]

    % ~   (2.iii) - Ogive on page 410. 
    % ~        * Median = 51
    % ~        * No. of students who failed = 46
    % ~        * No. of students who secured grade one = 14
    % ~       [MLQB - Page  - Q]
    \item Marks obtained by 200 students in an examination are given below: \hfill [5]

        \begin{table}[h]
        \centering
        \renewcommand{\arraystretch}{1.3}
        \begin{tabularx}{0.5\textwidth}{| p{2.75 cm} | X | }
            \hline
            \rowcolor{lightgray!30} Marks & Frequency \\
            \hline
            0 - 10 & 5 \\
            \hline
            10 - 20 & 11 \\
            \hline
            20 - 30 & 10 \\
            \hline
            30 - 40 & 20 \\
            \hline
            40 - 50 & 28 \\
            \hline
            50 - 60 & 37 \\
            \hline
            60 - 70 & 40 \\
            \hline
            70 - 80 & 29 \\
            \hline
            80 - 90 & 14 \\
            \hline
            90 - 100 & 6 \\
            \hline
        \end{tabularx}
        \end{table}

        Draw an ogive for the given distribution taking 1 cm = 10 marks on one axis
        and 1 cm = 20 students on the other axis. Using the graph, determine:

        \begin{itemize}
            \item The median marks
            \item The number of students who failed if minimum marks required to pass is 40
            \item If scoring 85 and more marks is considered as grade one, find the 
                number of students who secured grade one in the examination.
        \end{itemize}


\end{enumerate}
