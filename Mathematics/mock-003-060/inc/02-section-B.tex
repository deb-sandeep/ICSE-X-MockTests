% ~ Section B
\newpage
\begin{center}
   \large
   \textbf{SECTION B (20 marks)}\\
   \vspace{5mm}
   \normalsize
   \textit{(Attempt \textbf{any two} questions from this \textbf{Section})}
\end{center}
\par

% ======================================================================
% ~  Question 3
\noindent
\textbf{Question 3}
\begin{enumerate}[label=(\roman*)]

    % ~   (3.i) - x=-3 and y=-2 
    % ~       [MLQB - Page 147 - Q47]
    \item Given $\begin{bmatrix} x & 3 \\ y & 3 \end{bmatrix}$, \hfill [3] \\
        If $A^2 = 3I$, where $I$ is the identity matrix of order 2, find $x$ and $y$.

    % ~   (3.ii) - Proof 
    % ~       [MLQB - Page 362 - Q53]
    \item Prove that: \hfill [3] \\
        \[ \sqrt{\frac{1 - \cos \theta}{ 1 + \cos \theta}} = \mathrm{cosec} \ \theta - \cot \theta \]

    % ~   (3.iii) - {-4, -3, -2, -1, 0, 1, 2, 3, 4}
    % ~       [MLQB - Page 69 - Q40]
    \item Solve the following inequation, write down the solution set and represent
        it on the real number line: \hfill [4]
        \[-2 + 10x \leq 13x + 10 < 24 + 10x, x \in \mathbb{Z}\]

\end{enumerate}

% ======================================================================
% ~  Question 4
\noindent
\textbf{Question 4}
\begin{enumerate}[label=(\roman*)]

    % ~   (4.i) - n1 : n2 = 3:2 
    % ~       [MLQB - Page 398 - Q37]
    \item The average score of boys in an examination of a school is 71 and 
        of girls is 73. The average score of school in that examination is 
        71.8. Find the ration of the number of boys to the number of girls
        appeared in the examination. \hfill [3]

    % ~   (4.ii) - k = -2
    % ~       [MLQB - Page 132 - Q34]
    \item Using the Remainder Theorem find the remainders obtained when 
        $x^3 + (kx+8)x + k$ is divided by $x+1$ and $x-2$. \hfill [3]

    % ~   (4.iii) - X = [[-2, 5][3, 1]] 
    % ~       [MLQB - Page 145 - Q38]
    \item Given: \hfill [4]
        \[ 
            A = \begin{bmatrix} 2 & -6 \\ 2 & 0 \end{bmatrix},
            B = \begin{bmatrix} -3 & 2 \\ 4 & 0 \end{bmatrix},
            C = \begin{bmatrix} 4 & 0 \\ 0 & 2 \end{bmatrix}
        \]
        Determine the matrix $X$ such that: \[ A + 2X = 2B + C \]


\end{enumerate}

% ======================================================================
% ~  Question 5
\newpage
\noindent
\textbf{Question 5}
\begin{enumerate}[label=(\roman*)]

    % ~   (5.i) - {x:-13<x<23, x in R}
    % ~       [MLQB - Page 72 - Q49.iv]
    \item Find the solution set of the following inequalities and draw the graph
        of their solution sets: \hfill [4]
        \[ \abs{\frac{x-5}{3}} < 6, x \in \mathbb{R} \]

    % ~   (5.ii) - 22.214 
    % ~       [MLQB - Page 403 - Q57]
    \item Using the step deviation method find the arithmetic mean of the 
        distribution: \hfill [6]
        
        \begin{table}[h]
        \centering
        \renewcommand{\arraystretch}{1.3}
        \begin{tabularx}{0.75\textwidth}{| p {2.75 cm} | r | X | r | r | r | r | r | r | r | r | }
            \hline
            Variate (x) & 5 & 10 & 15 & 20 & 25 & 30 & 35 & 40 & 45 & 50 \\
            \hline
            Frequenty (f) & 20 & 43 & 75 & 67 & 72 & 45 & 39 & 9 & 8 & 6 \\
            \hline
        \end{tabularx}
        \end{table}


\end{enumerate}


