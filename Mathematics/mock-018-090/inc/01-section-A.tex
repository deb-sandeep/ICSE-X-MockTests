% ~ ........................................
% ~ Section A
% ~ ======================================== 
\begin{center}
   \large
   \textbf{SECTION A (20 marks)}\\
   \vspace{5mm}
   \normalsize
   \textit{(Attempt \textbf{all} questions from this \textbf{Section})}
\end{center}
\par
\noindent
% ~  Question 1
% ~  ----------
\textbf{Question 1}\\
Choose the correct answers to the questions from the given options. \hfill [10]\\
(Do not copy the questions, write the correct answers only.)
\par
\vspace{2mm}
\begin{enumerate}[label=(\roman*)]

    % ------- Question 1.1 ------------------------------------
    % ~   (i) - (a)
    % ~       [- 326/22]
    \item Find the capacity (in litres) of a conical vessel with radius 5 cm
        and slant height 13 cm. Take $\pi = \frac{22}{7}$.

        \begin{multicols}{2}
        \begin{enumerate}[label=(\alph*)]
            \item 0.314
            \item 0.267
            \item 3.14
            \item 314
        \end{enumerate}
        \end{multicols}

    % ------- Question 1.ii ----------------------------------
    % ~   (ii) - (b) 
    % ~       [EC - Page 328/18]
    \item The value of $\sin^2 \theta + \dfrac{1}{1 + \tan^2 \theta}$ is:

        \begin{multicols}{2}
        \begin{enumerate}[label=(\alph*)]
            \item 0 
            \item 1 
            \item 2 
            \item None of these
        \end{enumerate}
        \end{multicols}

    % ------- Question 1.iii ----------------------------------
    % ~   (iii) - (b)
    % ~       [EC - Page 382/7]
    \item The mean of first five prime numbers is: 

        \begin{multicols}{2}
        \begin{enumerate}[label=(\alph*)]
            \item 6.5 
            \item 5.6
            \item 4.5 
            \item 4.6
        \end{enumerate}
        \end{multicols}

    % ------- Question 1.iv ----------------------------------
    % ~   (iv) - (c)
    % ~       [EC - Page 400/1]
    \item In histogram, the width of the bars is proportional to:

        \begin{multicols}{2}
        \begin{enumerate}[label=(\alph*)]
            \item Frequency 
            \item Number of classes
            \item Class interval
            \item None of the above 
        \end{enumerate}
        \end{multicols}

    % ------- Question 1.v ----------------------------------
    % ~   (v) - (c)
    % ~       [EC - Page 302/5]
    \item If the height of a right circular cylinder is equal to the 
        diameter of the base, then its total surface area is:

        \begin{multicols}{2}
        \begin{enumerate}[label=(\alph*)]
            \item $\dfrac{4}{3} \pi h^2$
            \item $\dfrac{2}{3} \pi h^2$
            \item $\dfrac{3}{2} \pi h^2$
            \item $4 \pi h^2$
        \end{enumerate}
        \end{multicols}

    % ------- Question 1.vi ----------------------------------
    % ~   (vi) - (d)
    % ~       [EC - Page 160/11]
    \item Point P(a,b) is reflected in the x-axis to P'(5,-3). The 
        point P(a,b) is:

        \begin{multicols}{2}
        \begin{enumerate}[label=(\alph*)]
            \item $(-3,5)$
            \item $(-5,3)$
            \item $(-5,-3)$
            \item $(5,3)$
        \end{enumerate}
        \end{multicols}

    % ------- Question 1.vii ----------------------------------
    % ~   (vii) - (b)
    % ~       [EC - Page 324/3]
    \item If $\sin \theta + \sin^2 \theta = 1$, then $\cos^2 \theta + \cos^4 \theta = $

        \begin{multicols}{2}
        \begin{enumerate}[label=(\alph*)]
            \item $\dfrac12$ 
            \item 1
            \item 2
            \item 3
        \end{enumerate}
        \end{multicols}

    % ------- Question 1.viii ---------------------------------
    % ~   (viii) - (c)
    % ~       [EC - Page 382/6]
    \item If the mode of data 64, 60, 48, $x$, 43, 48, 43, 34 is 43, then $x+3$ is: 

        \begin{multicols}{2}
        \begin{enumerate}[label=(\alph*)]
            \item 44 
            \item 45 
            \item 46 
            \item 48 
        \end{enumerate}
        \end{multicols}

    % ------- Question 1.ix -----------------------------------
    % ~   (ix) - (c)
    % ~       [EC - Page c]
    \item If $\dfrac{5a}{7b} = \dfrac{4c}{3d}$, then by componendo and 
        dividendo:

        \begin{multicols}{2}
        \begin{enumerate}[label=(\alph*)]
            \item $\dfrac{5a + 7b}{5a - 7b} = \dfrac{4c - 3d}{4c + 3d}$ 
            \item $\dfrac{5a - 7b}{5a + 7b} = \dfrac{4c + 3d}{4c - 3d}$ 
            \item $\dfrac{5a + 7b}{5a - 7b} = \dfrac{4c + 3d}{4c - 3d}$ 
            \item $\dfrac{5a + 7b}{5a - 7b} = \dfrac{4c - 3d}{4c - 3d}$ 
        \end{enumerate}
        \end{multicols}

    % ------- Question 1.x ------------------------------------
    % ~   (x) - (c)
    % ~       [EC - Page 120/4]
    \item If a matrix has 4 elements, then which of the following cannot be the 
        order of the matrix.

        \begin{multicols}{2}
        \begin{enumerate}[label=(\alph*)]
            \item $2 \times 2$ 
            \item $1 \times 4$ 
            \item $2 \times 3$ 
            \item $4 \times 1$ 
        \end{enumerate}
        \end{multicols}

\end{enumerate}

% ======================================================================
% ~  Question 2
% ~  ----------
\par
\noindent
\textbf{Question 2}\\
\begin{enumerate}[label=(\roman*)]

    % ~   (2.i) - a=7 and b=6
    % ~       [EC - Page 113/35]
    \item Find the values of $a$ and $b$, if $(x-1)$ and $(x-2)$ 
        are factors of $x^3 - ax + b$. \hfill [3]

    % ~   (2.ii) - [[-5, 0], [11, -1]] 
    % ~       [EC - Page 127/33]
    \item Find the matrix $X$ such that $-A + 3B + X = 0$, where: \hfill [3]
        \[
            A = \begin{bmatrix*} -2 & 6  \\ 5 & 8 \end{bmatrix*}
            \text{ and }
            B = \begin{bmatrix*} 1 & 2  \\ -2 & 3 \end{bmatrix*}
        \]

    % ~   (2.iii) - 2059 
    % ~       [EC - Page 141/10]
    \item Given a G.P. with $a=729$ and 7\textsuperscript{th} 
        term 64, determing $S_7$. \hfill [4]

\end{enumerate}

