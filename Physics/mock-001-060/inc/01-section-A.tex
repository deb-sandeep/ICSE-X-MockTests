% ~ ........................................
% ~ Section A
% ~ ======================================== 
\begin{center}
   \large
   \textbf{SECTION A (20 marks)}\\
   \vspace{5mm}
   \normalsize
   \textit{(Attempt \textbf{all} questions from this \textbf{Section})}
\end{center}
\par
\noindent
% ~  Question 1
% ~  ----------
\textbf{Question 1}\\
Choose the correct answers to the questions from the given options. \hfill [7]\\
(Do not copy the questions, write the correct answers only.)
\par
\vspace{2mm}
\begin{enumerate}[label=(\roman*)]

    % ------- Question 1.i ----------------------------------
    % ~   (i) - (a) 
    % ~       [MLQB - Page 34 - Q 20]
    \item For a given mass and velocity the kinetic energy remains constant
        if \dots \dots

        \begin{enumerate}[label=(\alph*)]
            \setlength\itemsep{0em}
            \item mass is four times and velocity is $\frac12$
            \item mass is $\frac12$ and velocity is doubled
            \item mass and velocity are both squared
            \item mass is $\frac12$ and velocity is $\frac12$
        \end{enumerate}

    % ------- Question 1.ii ----------------------------------
    % ~   (ii) - (d) 
    % ~       [MLQB - Page 35 - Q 32]
    \item Which of the following is the correct expression for gain in 
        kinetic energy, if initial velocity is not zero?

        \begin{multicols}{2}
        \begin{enumerate}[label=(\alph*)]
            \item $k = \dfrac{1}{2}mv^2$
            \item $k = \dfrac{mv^2}{4}$
            \item $k = \dfrac{mv^2}{2t}$
            \item $k = \dfrac{1}{2}m(v^2-u^2)$
        \end{enumerate}
        \end{multicols}

    % ------- Question 1.iii ----------------------------------
    % ~   (iii) - (a)
    % ~       [MLQB - 23/21]
    \item A half meter scale is balanced at 5 cm mark when a 
        weight of 10 gf is suspended at one of its ends. The weight 
        of the scale is:

        \begin{multicols}{4}
        \begin{enumerate}[label=(\alph*)]
            \item 2.5 gf
            \item 100 gf
            \item 0.2 gf
            \item 50 gf
        \end{enumerate}
        \end{multicols}

    % ------- Question 1.iv ----------------------------------
    % ~   (iv) - (d)
    % ~       [MLQB - 22/7]
    \item The center of gravity of a regular object would lie \dots\dots

        \begin{enumerate}[label=(\alph*)]
            \setlength\itemsep{0em}
            \item On the body
            \item Outside the body
            \item Inside the body
            \item All of the above
        \end{enumerate}

    % ------- Question 1.v ----------------------------------
    % ~   (v) - (b)
    % ~       [MLQB - 1/1]
    \item Moment of force is \dots\dots to the distance of point of 
        application of force and fulcrum.

        \begin{enumerate}[label=(\alph*)]
            \item inversely proportional 
            \item directly proportional 
            \item not equal
            \item equal
        \end{enumerate}

    % ------- Question 1.vi ----------------------------------
    % ~   (vi) - (b)
    % ~       [MLQB - 51/16]
    \item In a block and tackle system of pulleys the load is lifted through a 
        height of 2 m while the effort is applied for 6 m. Hence the velocity
        ratio of this system is:

        \begin{multicols}{4}
        \begin{enumerate}[label=(\alph*)]
            \item 2
            \item 3
            \item 6
            \item 12
        \end{enumerate}
        \end{multicols}

    % ------- Question 1.vii ----------------------------------
    % ~   (vii) - (b)
    % ~       [MLQB - 52/25]
    \item A pulley system consisting of four pulleys has efficiency as 90\%.
        Calculate the Mechanical Advantage of the system.

        \begin{multicols}{4}
        \begin{enumerate}[label=(\alph*)]
            \item 360
            \item 3.6
            \item 36
            \item 0.36
        \end{enumerate}
        \end{multicols}

\end{enumerate}

% ======================================================================
% ~  Question 2
% ~  ----------
\newpage
\par
\noindent
\textbf{Question 2}\\
\begin{enumerate}[label=(\roman*)]

    % ~   (2.i) -
    % ~     a) [MLQB - 36/34]
    % ~        Zero
    % ~     b) [MLQB - 53/32]
    % ~        Definition of machine
    % ~     c) [MLQB - 38/67]
    % ~        Yes. Potential energy
    % ~        
    \item \hfill [3]
        \begin{enumerate}[label=(\alph*)]
            \item What is the work done by the gravitational force on 
                the Moon that revolves around the Earth?
            \item What is a machine?
            \item Is it possible that a body may possess energy even 
                when it is not in motion?
        \end{enumerate}

    % ~   (2.ii) - 
    % ~     a) [MLQB - 25/50]
    % ~        Centripetal force, towards the center
    % ~     b) [MLQB - 37/59]
    % ~        
    \item \hfill [2]
        \begin{enumerate}[label=(\alph*)]
            \item Name the force required for uniform circular motion. State 
                its direction.
            \item Mention one point of distinction between work and power.
        \end{enumerate}


    % ~   (2.iii) - 
    % ~     a) [MLQB - 38/66]
    % ~       Man moving along the slope. While moving along the 
    % ~       height, he works against gravity.
    % ~     b) [MLQB - 37/36]
    % ~       1J = 10^7 erg 
    % ~
    \item \hfill [2]
        \begin{enumerate}[label=(\alph*)]
            \item A man climbs a slope and another walks same distance 
                on level road. Which of the two expends more energy and 
                why?
            \item State the relation between S.I. and C.G.S. units of work.
        \end{enumerate}

    % ~   (2.iv) - [MLQB - 42/99]
    % ~     a) Work done = 600 J
    % ~     b) Gain in P.E. = 450 J
    % ~        
    \item A block of mass 30 kg is pulled up a slope, as shown in the 
        diagram with a constant speed, by applying a force of 200 N 
        parallel to the slope. A and B are initial and final positions 
        of the block. \hfill [2]

        \img{7cm}{2_iv.png}

        \begin{enumerate}[label=(\alph*)]
            \item Calculate the work done by force in moving the block
                from A to B.
            \item Calculate P.E. gained by block.
        \end{enumerate}
        [Take g = 10 ms\textsuperscript{-2}]

    % ~   (2.v) - 
    % ~     a) [MLQB - 38/70]
    % ~       Potential energy of the bow string 
    % ~     b) [MLQB - 25/47.ii]
    % ~       Intersection of medians
    % ~        
    \newpage
    \item \hfill [2]
        \begin{enumerate}[label=(\alph*)]
            \item When an arrow is shot from its bow, it has kinetic energy.
                From where does it get the kinetic energy?
            \item Where is the C.G. of a triangular lamina located? 
        \end{enumerate}

    % ~   (2.vi) - 
    % ~     a) [MLQB - 37/49]
    % ~        Energy, 1 eV = 1.6 x 10^{-19} J
    % ~     b) [MLQB - 38/81]
    % ~        Body of mass 50 kg will have more K.E.
    % ~        
    \item \hfill [2]
        \begin{enumerate}[label=(\alph*)]
            \item What physical quantity does electron volt (eV) measure?
                How is it related to S.I. unit of that quantity?
            \item A body of mass 50 kg and another body of mass 100 kg 
                have equal momentum. Which body will have more kinetic 
                energy?
        \end{enumerate}

\end{enumerate}

