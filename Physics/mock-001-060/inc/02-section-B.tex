% ~ ........................................
% ~ Section B
% ~ ======================================== 
\par\noindent\rule{\textwidth}{0.4pt}
\begin{center}
   \large
   \textbf{SECTION B (20 marks)}\\
   \vspace{5mm}
   \normalsize
   \textit{(Attempt \textbf{any two} questions from this \textbf{Section})}
\end{center}
\par

% ======================================================================
% ~  Question 3
% ~  ----------
\noindent
\textbf{Question 3}
\begin{enumerate}[label=(\roman*)]

    % ~   (3.i) - 
    % ~       [MLQB - 40/90]
    % ~         Work done by a source of power 1kW in one hour
    % ~       [MLQB - ]
    % ~         1kWh = 3.6x10^6J
    % ~       [MLQB - ]
    % ~         Product of the magnide of force and the perpendicular
    % ~         distance of the line of action of force from axis 
    % ~         of rotation.
    % ~
    \item \hfill [3]
        \begin{enumerate}[label=(\alph*)]
            \item Define kilowatt hour.
            \item How is kilowatt hour related to joule?
            \item Define moment of force. 
        \end{enumerate}

    % ~   (3.ii) - t=500 sec
    % ~       [MLQB - 43/106]
    \item How long should an electric motor of power 2 H.P. operate, so 
        as to pump 5 m\textsuperscript{3} of water from a depth of 
        15 m. \hfill [3]

        [Take g = 10 N kg\textsuperscript{-1}]

    % ~   (3.iii) - 
    % ~       [MLQB - 54/42]
    % ~       * Make the lower block as light as possible
    % ~       * Reduce friction in the pulleys
    % ~       [MLQB - 37/60]
    % ~       * Electrical to sound 
    % ~       * Electrical to heat and light
    \item \hfill [4]
        \begin{enumerate}[label=(\alph*)]
            \item State two ways of increasing the efficiency of a block 
                and tacke system of pulleys.
            \item State the energy changes in the following devices:
                \begin{enumerate}
                    \setlength\itemsep{0em}
                    \item A loudspeaker
                    \item A glowing electric bulb
                \end{enumerate}
        \end{enumerate}

\end{enumerate}

% ======================================================================
% ~  Question 4
% ~  ----------
\newpage
\noindent
\textbf{Question 4}
\begin{enumerate}[label=(\roman*)]

    % ~   (4.i) - 
    % ~       [MLQB - Page 40 - Q 89]
    \item Give one example when: \hfill [3]
        \begin{itemize}
            \setlength\itemsep{0em}
            \item Heat energy changes to kinetic energy.
            \item Sound energy changes to electrical energy.
            \item Light energy changes to chemical energy.
        \end{itemize}

    % ~   (4.ii) - 
    % ~       [MLQB - 25/39]
    % ~       [MLQB - 25/41]
    \item \hfill [3]
        \begin{enumerate}[label=(\alph*)]
            \item State two conditions for a body to be in equilibrium.
            \item Give an example of static equilibrium.
        \end{enumerate}

    % ~   (4.iii) - 52.5 gf
    % ~       [MLQB - 28/73]
    \item A uniform meter scale is balanced at 40 cm mark, when 
        weights of 20 gf and 5 gf are suspended at 5 cm mark and 
        75 cm mark respectively. Calculate the weight of meter 
        scale. \hfill [4]

\end{enumerate}

% ======================================================================
% ~  Question 5
% ~  ----------
\noindent
\textbf{Question 5}
\begin{enumerate}[label=(\roman*)]

    % ~   (5.i) - F = 50N, Power = 1000 W
    % ~       [MLQB - Page 43 - Q 102]
    \item An energy of 4000 J causes a displacement of 80 m in 4 s. 
        Calculate (i) Force (ii) Power. \hfill [2]

    % ~   (5.ii) - Tension is string is same. # strands pulling 
    % ~       the load increases. Hence MA increases.
    % ~       [MLQB - 54/41]
    \item In the case of block and tackle arrangement, the mechanical 
        advantage increases with the number of pulleys. Explain. \hfill [2]

    % ~   (5.iii) - 
    % ~     a) 10 J
    % ~     b) 10 J
    % ~     c) 10 m/s
    % ~       [MLQB - Page  - Q]
    \item The figure below shows a simple pendulum of mass 200 g. It is 
        displaced from the mean position A to the extreme position B. 
        The potential energy at the position A is zero. At the position 
        B the pendulum bob is raised by 5 cm. \hfill [6]

        \img{6cm}{5_iii.png}

        \begin{enumerate}[label=(\alph*)]
            \item What is the potential energy of the pendulum at the 
                position B?
            \item What is the total mechanical energy at point C?
            \item What is the speed of the bob at the position A when 
                released from B?
        \end{enumerate}
        [Take g = 10 ms\textsuperscript{-2} and no loss of energy]


\end{enumerate}

